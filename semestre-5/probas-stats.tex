\documentclass[a4paper,10pt,french,openany]{memoir}
\usepackage[utf8]{inputenc}
\usepackage{babel}

\usepackage{clovisai}
\newcommand{\Proba}{\mathbb{P}}
\newcommand{\Esper}{\mathbb{E}}
\newcommand{\tribu}[1]{\mathcal{#1}}
\newcommand{\normale}{\mathcal{N}}
\newcommand{\uniforme}{\mathcal{U}}
\newcommand{\cauchy}{\mathcal{C}}
\newcommand{\expo}{\mathcal{E}}
\newcommand{\binomiale}{\mathcal{B}}
\newcommand{\implique}{\Rightarrow}
\newcommand{\abs}[1]{\lvert #1 \rvert}
\newcommand{\laweq}{\overset{\mathcal L}=}
\newcommand{\xonen}{x_1\dots x_n}
\newcommand{\Xonen}{X_1\dots X_n}
\newcommand{\xoned}{x_1\dots x_d}
\newcommand{\Xoned}{X_1\dots X_d}
\newcommand{\xonek}{x_1\dots x_k}
\DeclareMathOperator{\jac}{Jac}
\DeclareMathOperator{\Var}{Var}
\DeclareMathOperator{\Cov}{Cov}

%opening
\title{Probabilités et Statistiques}
\author{François Dufour (\hrefu{mailto://francois.dufour@math.u-bordeaux.fr}{francois.dufour@math.u-bordeaux.fr})\\Notes prises par Ivan Canet}

\begin{document}

\maketitle
\tableofcontents

\chapter{Probabilités sur un espace fini}
\section{Probabilités sur un espace fini, événements}
\subsection{Définitions}

\paragraph{Notation}
On s'intéresse à l'expérience \emph{aléatoire} qui conduit à la réalisation d'un seul résultat parmi un nombre \emph{fini} de résultats possibles, notés $\omega_1\dots \omega_n$. On note $\Omega = \lbrace \omega_1\dots\omega_n \rbrace$ l'ensemble de ces résultats.\\
Par exemple, un jeu de pile ou face peut être modélisé en $\Omega = \lbrace P, F \rbrace$.

\paragraph{Probabilité}
Une probabilité $\Proba$ sur l'ensemble $\Omega = \lbrace \omega_1\dots\omega_n \rbrace$ est une \emph{pondération} de $p_1\dots p_n$ où $n$ est un nombre réel positif, tels que:
\[\sum_{i=1}^n p_i = 1\]

\paragraph{Événement}
On appelle événement tout sous-ensemble de $\Omega$.

Pour un événement $A \subset \Omega$, on définit:
\[\Proba(A) = \sum_{i=1}^n p_i\]
où $\Proba(A)$ est appelé la \emph{probabilité de $A$}.

\paragraph{Terminologie}
\begin{itemize}
 \item Si $\Proba(A)=0$, l'événement est dit \emph{négligeable}.
 \item Si $\Proba(A)=1$, l'événement est dit \emph{presque sûr}.
 \item Le contraire de $A$ est $\bar{A}$.
 \item L'événement «~$A$ et $B$~»: $A \inter B$
 \item L'évenement «~$A$ ou $B$~»: $A \union B$
\end{itemize}

On peut démontrer que:
\[ \Proba(A \union B) = \Proba(A) + \Proba(B) - \Proba(A \inter B) \]
\[ \Proba(A \inter B) = \Proba(A) + \Proba(B) - \Proba(A \union B) \]

\paragraph{Fonction indicatrice} La fonction indicatrice d'un événement $A$ est définie comme
$ I_A: \Omega \mapsto \lbrace 0, 1 \rbrace $
telle que:
\[
\forall u \in \Omega, \; I_A(\omega) = \begin{cases*}
1 & si $\omega \in A$ \\ 
0 & si $\omega \notin A$
\end{cases*}
\]

On retrouve aussi
\[I_{A \inter B} = I_A I_B\]
\[I_{A \union B} = I_A + I_B - I_{A \inter B}\]

\subsection{Exemple}

La probabilité uniforme correspond à ce que tous les événements élémentaires $\omega_1\dots\omega_n$ aient la même probabilité.

\[ p_i = \frac1n, \; \forall i \in \lbrace 1 \dots n \rbrace \]

\section{Probabilités conditionnelles et indépendance}
\subsection{Probabilité conditionnelle}

La probabilité conditionnelle permet de prendre en compte l'information dont on dispose pour actualiser la probabilité d'un événement.

\paragraph{Définition}
Soit $\Omega$ un ensemble muni d'une probabilité $\Proba$. On considère deux événements $A$ et $B$. La probabilité conditionnelle de $A$ sachant $B$ est définie par:
\[
\Proba(A \sachant B) =
    \begin{cases*}
        \frac{\Proba(A \inter B)}{\Proba(B)} & si $\Proba(B) > 0$ \\
        \Proba(A) & sinon
    \end{cases*}
\]

\paragraph{Exemple 1}
Quelle est la probabilité qu'un individu ayant deux enfants ait un garçon, sachant qu'il a une fille?

$\Omega = \lbrace (F,F), (G,G), (F,G), (G,F) \rbrace$

On définit $A$ l'événement \emph{avoir un garçon} et $B$ l'événement \emph{avoir une fille}.

\[ \Proba(A \sachant B) = \frac{\Proba(A \inter B)}{\Proba(B)} = \frac{card(A \inter B)}{card(B)} = \frac{2}{3} \]
donc, la probabilité que l'individu ait un garçon, en sachant qu'il a une fille, est $\frac 2 3$.

\paragraph{Exemple 2}
Quelle est la probabilité qu'un individu ayant deux enfants ait un garçon, sachant que l'aînée est une fille?

$\Omega = \lbrace (F,F), (G,G), (G,F) \rbrace$ \textit{(on ignore l'ordre)}

\[ \Proba(F,F)=\Proba(G,G)=\frac 1 4 \]
\[ \Proba(F,G)=\frac 1 2 \]
donc, la probabilité que l'individu ait un garçon, en sachant que l'aînée est une fille, est $\frac 1 2$.

\paragraph{Remarque}
$\Proba(A \inter B) = \Proba(A \sachant B) \Proba(B)$

\paragraph{Formule de Bayes}
Soit $B_1 \dots B_n$ une partition de $\Omega$ ($B_i \inter B_j = \emptyset$ pour $i \neq j$ et $\bigcup_{i=1} B_i = \Omega$) et $A \subset \Omega$ ($\Proba(A) > 0$).

\[
\Proba(B_i \sachant A) = \frac{\Proba(A \sachant B_i) \Proba(B_i)}{\sum_{j=1}^n \Proba(A \sachant B_j) \Proba(B_j)} = \frac{\Proba(A \inter B_i)}{\sum_{j=1}^n \Proba(A \inter B_j)}
\]

\subsection{Indépendance}

\paragraph{Définition}
Soit $\Omega$ muni d'une probabilité $\Proba$, $A$ et $B$ sont des événements indépendants si et seulement si:
\[\Proba(A \inter B) = \Proba(A) \Proba(B)\]
ou, de façon équivalente:
\[\Proba(A \sachant B) = \Proba(A)\]
\[\Proba(B \sachant A) = \Proba(B)\]

$A_1 \dots A_n$ sont indépendants si et seulement si:
\[\forall J \subset \lbrace 1 \dots n \rbrace, \; \Proba(\Inter_{j \in J} A_j) = \prod_{j \in J} \Proba(A_j)\]

\chapter{Variables aléatoires discrètes}

\section{Espace de probabilité}

\paragraph{Définition}
Une tribu $\tribu{A}$ sur $\Omega$ est une famille de sous-ensembles d'$\Omega$ qui vérifie les propriétés suivantes:

\begin{itemize}
 \item $\emptyset$ et $\Omega$ sont des éléments de $\tribu A$.
 \item $A \in \tribu A \implique \bar A \in \tribu A $
 \item Si $(A_i)_{i \in \setN} \subset \tribu A$, alors $\Union_{i=1}^{+\infty} A_i \in \tribu A$
\end{itemize}

\paragraph{Exemples}
\begin{itemize}
 \item $\{\emptyset, \Omega\}$ est une tribu,
 \item $P(\Omega)$ est l'ensemble des parties de $\Omega$, c'est une tribu,
 \item $A \in \Omega$, $\{ A, \bar A, \emptyset, \Omega \}$ est une tribu.
\end{itemize}

\paragraph{Définition}
Soit $\Omega$ muni d'une tribu notée $\tribu A$.
On appelle probabilité sur $(\Omega, \tribu A)$ une application 
\( \Proba: \tribu A \mapsto [0, 1] \)
qui vérifie:
\begin{itemize}
 \item $\Proba(\Omega) = 1$
 \item Si $(A_i)_{i \in I}$ est une famille dénombrable d'éléments de $\tribu A$ \emph{deux à deux disjoints} ($A_i \inter A_j = \emptyset$ pour $i \neq j$), alors:
 \[ \Proba(\Union_{i\in I} A_i) = \sum_{i\in I} \Proba(A_i) \]
\end{itemize}

Tout ce qui a été introduit au chapitre précédent reste vrai:

\[ \Proba(A \union B) = \Proba(A) + \Proba(B) - \Proba(A \inter B) \]

\[ \text{Si } \Proba(B) > 0 \text{ alors } \Proba(A \sachant B) = \frac{\Proba(A \inter B)}{\Proba(B)} \]

\paragraph{Indépendance}
Une famille quelconque d'événements est dite indépendante si toute sous-famille \emph{finie} est indépendante.

\section{Variables aléatoires discrètes}

\subsection{Définition}
On appelle variable aléatoire discrète $X$ une application $X: \Omega \mapsto F$ où $F$ est un ensemble fini ou dénombrable. Pour $x \in F$ on note:
\[ \{X=x\} = \{\omega \in \Omega: X(\omega) \in F\} \]

La famille de nombres $\Proba(X=x)_{x \in F}$ s'appelle la loi de $X$.

Si $A \in \Omega$ alors la fonction indicatrice $I_A$ de $A$ est une variable aléatoire et $\Proba(I_A = 1) = \Proba(A)$.

\[I_A : \Omega \mapsto \{0,1\}\]

\begin{align*}
 \{I_A=1\} &= \{\omega \in \Omega: I_A(\omega)=1\}\\
           &= A
\end{align*}

\subsection{Indépendance des variables aléatoires}

\paragraph{Définition}
Deux variables aléatoires discrètes $X$ et $Y$ à valeurs dans $F$ et $G$ respectivement sont dites indépendantes si:
\[ \Proba(X=x, Y=y)=\Proba(X=x) \Proba(Y=y) \]
où $x\in F$ et $y \in G$, où l'on note:
\[ \{X=x, Y=y\}=\{X=x\}\inter\{Y=y\} \]

\paragraph{Définition}
$n$ variables aléatoires sont indépendantes si pour tout $(x_i)$, \[\Proba(X_1=x_1, X_2=x_2\dots X_n=x_n) = \prod \Proba(X_i=x_i)\]
c'est à dire, que leurs intersections sont nulles.

\subsection{Lois discrètes usuelles}

Ce sont des variables aléatoires à valeur dans $F \subset \setN$.

\paragraph{Loi de Bernouilli} de paramètre $p \in [0,1]$.

\begin{cquote}{}
Je joue à pile ou face.

$\Proba(X=1) = 0.1$ signifie que j'ai 10\% de chances de gagner.
\end{cquote}

On considère $X: \Omega \mapsto \{0,1\}$:
\[ \Proba(X=1)=p \qquad\qquad \Proba(X=0)=1-p \]

L'événement $\{X=1\}$ représente un succès, et sa probabilité est associée à $p$.

\paragraph{Loi géométrique} de paramètre $p \in ]0,1]$.

\begin{cquote}{}
Si je joue beaucoup de fois à pile ou face, au bout de la quantième vais-je gagner?

$\Proba(X=5)=0.1$ signifie que j'ai 10\% de chances de gagner lors de 5\ieme{} tentative.
\end{cquote}

$X:\Omega \mapsto \setN$ suit une loi géométrique si:
\[ \Proba(X=k)=(1-p)^{k-1}p \]
où $k \in \setNs$, et $p$ représente la probabilité de gagner.

Autrement, on considère des variables aléatoires $\{Y_i\}_{i\in \setNs}$ indépendantes, et $Y_i$ suit une loi de Bernouilli de paramètre $p\in ]0,1]$.
\[ \Proba(Y_i = 1) = p \qquad\qquad \Proba(Y_i=0)=1-p \]
où $i$ représente l'instant du jeu.

\paragraph{Loi binomiale} de paramètres $n\in \setNs$ et $p\in [0,1]$.

\begin{cquote}{}
Je joue à pile ou face, combien de fois je peux m'attendre à gagner?

$\Proba(X=5)=0.1$ avec $X\leadsto\binomiale(15, p)$ signifie que j'ai 10\% de chances de gagner 5 fois à pile ou face, si je joue 15 parties.
\end{cquote}

C'est la loi variable $X:\Omega \mapsto \{0,n\}$ définie par: 
\[\Proba(X=k)=\parmi k n p^k (1-p)^{p-k}\]
où $k \in \{0,n\}$.

$X$ correspond au nombre de succès lors de la répétition indépendante de jeux dont la probabilité de succès vaut $p$.

On considère que $n$ variables aléatoires $(Y_i)_{i\in 1\dots n}$ sont indépendants si $Y_i$ suit une loi de Bernouilli de paramètre $p$.

\paragraph{Loi de Poisson} de paramètre $\lambda \in \setRp$.
\begin{cquote}{}
Combien de personnes vont entrer dans un magasin dans la journée?

$\Proba(X=5)=0.1$ signifie qu'il y a 10\% de chances que 5 personnes entrent dans le magasin aujourd'hui.
\end{cquote}

C'est la loi variable aléatoire $X: \Omega \mapsto \setN$ définie par:
\[\Proba(X=k)=\frac{\lambda^k}{k!} e^{-\lambda}\]

\subsection{Loi marginale}

On considère deux variables aléatoires discrètes $X$ et $Y$ à valeurs dans $F$ et $G$.
\begin{equation} \Proba(X=x)_{x\in F} \qquad\qquad \Proba(Y=y)_{y\in G} \label{eq:couple-solo}\end{equation}

$(X,Y)$ est une variable aléatoire à valeur dans $F \times G$ et est discrète, donc:
\begin{equation} \Proba((X,Y)=(x,y))_{(x,y) \in F \times G} \label{eq:couple-duo}\end{equation}

\paragraph{Démonstration}
\begin{align*}
 \Union_{y\in G} \{(X,Y)=(x,y)\} &= \Union_{y\in G} (\{X=x\}\inter\{Y=y\})\\
    &= \{X=x\}\inter \Omega\\
    &= \{X=x\}
\end{align*}

\begin{align*}
 \Proba(X=x) &= \Proba(\Union_{y\in G} \{X=x,Y=y\}) \\
    &= \sum_{y\in G} \Proba((X,Y)=(x,y))
\end{align*}

Si on connait la loi du couple~(\ref{eq:couple-duo}), on peut calculer chacune de ses composantes~(\ref{eq:couple-solo}). L'inverse n'est généralement vrai (ce n'est possible que si $X$ et $Y$ sont indépendants).

\section{Espérance et variance}

\subsection{Espérance}

\paragraph{Définition}
Soit $X$ une variable aléatoire discrète à valeur $F\subset \setR$. On va dire que $X$ est intégrable si:
\[\sum_{x\in F} \lvert x \rvert \Proba(X=x) < +\infty \]

Dans ce cas, l'espérance de $X$ est notée $\Esper[X]$ et vaut:
\[ \Esper[X]=\sum_{x\in F} x\, \Proba(X=x) \]

\paragraph{Cas particuliers}
L'espérance de $X$ dans le cas où $X$ suit une loi:
\begin{itemize}
 \item De Bernouilli de paramètre $p$: \[ \Esper[X]=p \]
 \item Poisson de paramètre $\lambda \in \setR$: \[\Esper[X]=\lambda\]
 \item Géométrique de paramètre $p\in ]0,1]$: \[ \Esper[X]=\frac 1 p \]
\end{itemize}

\paragraph{Propriétés}
\begin{itemize}
 \item Linéarité: Si $X$ et $Y$ sont des variables aléatoires discrètes à valeurs réelles intégrables et $\lambda \in \setR$, alors $X + \lambda Y$ est intégrable et: \[\Esper[X+\lambda Y] = \Esper[X] + \lambda\Esper[Y]\]
 \item Positivité: Si $X$ est une variable aléatoire discrète intégrable et presque sûrement positive (ie. $\Proba(X \geq 0) = 1$), alors: \[\Esper[X] \geq 0 \] \[\Esper[X] = 0 \;\Rightarrow\; \Proba(X=0)=1 \text{ et } X=0 \text{ presque sûrement}\]
 \item Si $X$ et $Y$ sont deux variables aléatoires intégrables: \[\Proba(X \leq Y) = 1 \;\Rightarrow\; \Esper[X]\leq\Esper[Y]\]
 \item Condition suffisante d'intégrabilité: Si $Y$ est une loi discrète presque sûrement positive et intégrable, $\Proba(\abs{X}\leq Y) = 1$ implique que $X$ est intégrable.
\end{itemize}

\paragraph{Théorème de transfert}
Soit $X:\Omega \mapsto F$ une variable aléatoire discrète et $f:F \mapsto \setR$, alors $f(X)$ est une variable discrète et intégrable si: \[\sum_{x\in F} \abs{f(x)} \,\Proba(X=x) < \infty\]

Dans ce cas, on a: \[\Esper[f(X)] = \sum_{x\in F} f(x) \,\Proba(X=x)\]

\paragraph{Remarque}
Si $f: F\mapsto\setR$ est bornée, alors $f(x)$ est intégrable; si $\abs{f(x)} \leq M$ $\forall x \in F$, alors: \[\sum_{x\in F} \abs{f(x)} \Proba(X=x) \leq M < \infty\]

\paragraph{Exemple}
Calcul de l'espérance de la loi binomiale de paramètres $n\in\setNs$ et $p\in[0,1]$: $\Esper[X]=n p$.

\paragraph{Proposition}
Si $X$ et $Y$ sont des variables aléatoires discrètes indépendantes à valeur dans $F$ et $G$ respectivement, alors pour toutes fonctions $f: F\mapsto \setR$ et $g: G\mapsto \setR$ avec $f(X)$ et $g(Y)$ intégrables, on a que $f(X) g(Y)$ est intégrable et:\[\Esper[f(X) g(Y)]=\Esper[f(X)] \,\Esper[g(Y)]\]

\paragraph{Proposition}
Si $X$ et $Y$ sont deux variables aléatoires discrètes à valeur dans $F$ et $G$ et vérifient \[\Esper[f(X) g(Y)] = \Esper[f(X)]\,\Esper[g(Y)]\] pour toutes fonctions $f:F\mapsto \setR$ et $g: G\mapsto \setR$ bornées, alors $X$ et $Y$ sont indépendantes.

\subsection{Variance}

\paragraph{Définition}
Soit $X:\Omega \mapsto F$ une variable aléatoire discrète à valeurs dans $\setR$, $X$ est dite de ``carré intégrable'' si $\sum_{x\in F}x^2\, \Proba(X=x) < \infty$ et, dans ce cas, la variance vaut: \[\Var(x)=\Esper[(X-\Esper[X])^2]\]

La racine carrée de la variance s'appelle \emph{l'écart-type}, et se note: \[\sigma(x)=\sqrt{\Var(x)}\]

\paragraph{Remarques}
\begin{itemize}
 \item Si $X$ est de carré intégrable, alors $X$ est intégrable. En effet, $\abs{X}\leq\frac{1+X^2}2$ et d'après la condition suffisante d'intégrabilité, on déduit que $X$ est intégrable; d'où:
 \begin{align*}
 \Esper[X] &= \sum_{x\in F} x\,\Proba(X=x) \\
 X^2-2 \Esper[X] X+\Esper[X]^2 &= (X-\Esper[X])^2
 \end{align*}
 donc la variance a bien un sens.
 \item $\Var(X)=\Esper[X^2]-(\Esper[X])^2$
 \item $\Esper[X^2]\geq\Esper[X]^2$
\end{itemize}

\paragraph{Proposition}
Soient $X_1\dots X_n$ des variables aléatoires de carré intégrable indépendantes, alors $\sum_{i=1}^n X_i$ est de carré intégrable, et on a:
\[\Var \left(\sum_{i=1}^n X_i\right) = \sum_{i=1}^n \Var(X_i)\]

\paragraph{Démonstration}
\begin{align*}
 \Var \left(\sum_{i=1}^n X_i\right) &= \Esper\left[\left(\sum_{i=1}^n X_i - \Esper\left[\sum_{i=1}^n X_i\right]\right)\right] \\
    &= \Esper\left[\left(\sum_{i=1}^n (X_i - \Esper[X_i])\right)\right] \\
    &= \Esper\left[\sum_{i=1}^n (X_i - \Esper[X_i]) \sum_{i=1}^n (X_j - \Esper[X_j])\right] \\
    &= \sum_{i=1}^n \sum_{j=1}^n \Esper[(X_i - \Esper[X_i])\,(X_j - \Esper[X_j])] \\
    &= \sum_{i=1}^n \Esper[(X_i - \Esper[X_i])^2] + \sum_{i=1}^n \sum_{\genfrac{}{}{0pt}{}{j=1}{i\neq j}}^n \Esper[(X_i - \Esper[X_i])(X_j - \Esper[X_j])]
\end{align*}

On pose $\varphi_i(x)$ telle que:
\[\varphi_i(x) = x - \Esper[X_i]\]
d'où:
\[\Esper[\varphi_i(X_i) \varphi_j(X_j)] = \Esper[\varphi_i(X_i)] \, \Esper[\varphi_j(X_j)] = 0\]

\section{Fonctions génératrices}

\paragraph{Définition}
Soit $X: \Omega \mapsto \setN$ une variable aléatoire à valeurs entières. On appelle fonction génératrice la fonction $g_X:[-1,1]\mapsto\setR$ telle que:
\[g_X(s)=\Esper[s^X]=\sum_{k\in \setN} s^k\,\Proba(X=k)\]

\paragraph{Proposition}
La fonction génératrice d'une variable aléatoire à valeurs entières caractérise sa loi.
\[ X \laweq Y \:\Leftrightarrow\: g_X(s)=g_Y(s), \quad\forall s\in [-1,1] \]
où «~$\laweq$~» signifie ``sont de même loi''.

\paragraph{Démonstration}
On cherche à retrouver $\Proba(X=n)$ pour $n\in \setN$:
\[ \Proba(X=n)=\frac{g_X^{(n)}(0)}{n!} \]
où $g^{(n)}$ représente la dérivée $n$-ième d'une fonction.

\paragraph{Exemple}
\begin{itemize}
 \item Loi binomiale de paramètres $n\in\setNs$ et $p\in[0,1]$:
 \begin{align*}
    g_X(s) &= \sum_{k=0}^n s^k\,\Proba(X=k) \\
    &= \sum_{k=0}^n s^k\parmi{k}{n} p^k (1-p)^{n-k} \\
    &= (p s + 1 - p)^n
 \end{align*}
 \item Loi géométrique de paramètre $p\in[0,1]$:
 \begin{align*}
    g_X(s) &= \sum_{k\in\setNs} s^k p (1-p)^{k-1} \\
    &= p s \sum_{k\in\setN} (s (1-p))^k \\
    &= \frac{p s}{1 - s(1-p)}
 \end{align*}
 \item Loi de Poisson de paramètre $\lambda$:
 
 \begin{align*}
    g_X(s) &= \sum_{k\in\setN} s^k \frac{\lambda^k}{k!} e^{-\lambda} \\
    &= e^{-\lambda} \sum_{k\in\setN} \frac{(\lambda s)^k}{k!} \\
    &= e^{-\lambda} e^{\lambda s} \\
    &= e^{\lambda(s-1)}
 \end{align*}
\end{itemize}

\paragraph{Proposition}
Si $X$ et $Y$ sont deux variables aléatoires à valeurs entières indpendantes:
\[g_{X+Y}(s)=g_X(s) g_Y(s)\]

\paragraph{Exemple}
On considère $(X_i)_{i\geq 1}$ une suite de variables aléatoires à valeurs entières \emph{indépendantes et identiquement distribuée} (IID.) et $N$ une variable aléatoire à valeurs entières indépendante de la suite $(X_i)_{i\geq 1}$. On pose:
\[S=\begin{cases*}
X_1+X_2+\dots+X_N & si $N\neq0$ \\ 
0 & si $N=0$
\end{cases*}\]

Si $X_i$ et $N$ suivent des lois géométriques de paramètre $q$, quelle est la loi de~$S$?

\[
 g_S(s) = \Esper[s^S] = \sum_{k=0}^\infty s^k \Proba(S=k)
\]

Réécrivons $\Proba(S=k)$:
\begin{align*}
 \Proba(S=k) &= \sum_{n=1}^\infty \Proba(N=n)\,\Proba(S=k \sachant N = n)\\
 &= \sum_{n=1}^\infty q (1-q)^{n-1} \,\Proba(S=k\sachant N=n)
\end{align*}

Donc:
\begin{align*}
 g_S(s) &= \sum_{k=0}^\infty s^k \sum_{n=1}^\infty q (1-q)^{n-1} \,\Proba(S=k\sachant N=n) \\
 &= \sum_{n=1}^\infty q (1-q)^{n-1} \sum_{k=0}^\infty s^k \,\Proba(S=k\sachant N=n)
\end{align*}

On reconnaît la somme de $n$ variables indépendantes, on peut donc utiliser \[g_{X_1} + g_{X_2} + \dots + g_{X_n} (s) = \prod_{i=1}^n g_{X_i} (s)\]

Donc:
\begin{align*}
 g_S(s) &= \sum_{n=1}^\infty q (1-q)^{n-1} g_X(s)^n\\
 &= q g_X(s) \sum_{n=0}^\infty (1-q)^n g_X (s)^n\\
 &= \frac{q g_X(s)}{(1-(1-q) g_X(s))}
\end{align*}

\chapter{Variables aléatoires à densité}

\section{Variables aléatoires réelles à densité}

\paragraph{Définition}
Soit $(\Omega, \tribu A, \Proba)$ un espace de probabilité. On dit que $X:\Omega\mapsto\setR$ est une variable aléatoire à densité $p:\setR\mapsto\setRp$ si:
\[\Proba(\omega\in\Omega: X_\omega \in \;]a,b])=\int_a^b p(x) \dif x\]
où $a$ et $b$ sont des éléments de $\setR\union\lbrace-\infty,+\infty\rbrace$, et $a<b$.

\paragraph{Remarque}
$\Proba(X\in\setR)=1$ par définition, d'où $\int_{\setR} p(x) \dif x = 1$. De plus, $\forall x \in \Omega$, $\Proba(X=x)=0$ car:
\begin{align*}
    \Proba(X=x) &= \lim_{n\to\infty} \int_{x-\frac1n}^x p(y) \dif y \\
    &= \lim_{n\to\infty} \Proba\left(X\in\left]x-\frac1n,x\right]\right)
\end{align*}
et on peut en déduire que:
\begin{align*}
    \Proba(X\in\;]a,b]) &= \Proba(X\in[a,b]) \\
    &= \Proba(X\in[a,b[) \\
    &= \Proba(X\in\;]a,b[)
\end{align*}

\paragraph{Exemples de densités}
\begin{itemize}
 \item $X$ suit la loi uniforme de l'intervalle $[a,b]$ où $a<b$ et on note $X\leadsto\uniforme([a,b])$ si $X$ possède la densité \[p(x)=\frac1{b-a}I_{[a,b]}(x)\]
 
 \item Loi exponentielle de paramètre $\lambda>0$: on note $X\leadsto\expo(\lambda)$ si \[p(x)=\lambda e^{-\lambda x} I_{\setRp}(x)\]
 
 \item Loi gaussienne (normale) de paramètres $\mu$ et $\sigma^2>0$: on note $X\leadsto\normale(\mu,\sigma^2)$: \[p(x)=\frac1{\sqrt{2\pi}} e^{-\frac{(x-\mu)^2}{2\sigma^2}}\]
 
 \item Loi de Cauchy de paramètre $a>0$: on note $X\leadsto\cauchy(x)$ si \[p(x)=\frac{a}{\pi(x^2+a^2)}\]
\end{itemize}

\section{Espérance et variance}

\paragraph{Espérance}
La variable aléatoire $X:\Omega\mapsto\setR$ qui possède la densité $p$ est dite:
\begin{itemize}
 \item intégrable si \[\int_{\setR} \abs{x}\,p(x) \dif x < \infty\]
 son espérance vaut alors: \[\Esper[X] = \int_{\setR} x\,p(x) \dif x\]
 
 \item de carré intégrable si \[\int_{\setR} x^2\,p(x) \dif x < \infty\]
 on a alors aussi: \[\Esper[X^2] = \int_{\setR} x^2\,p(x)\dif x\]
\end{itemize}

\paragraph{Variance}
La variance vaut: \[\Var(X)=\Esper[X^2]-\Esper[X]^2\]

\paragraph{Propriétés de l'espérance}
\begin{itemize}
 \item Linéarité si $X$ et $Y$ sont à densité et intégrables; \[\Esper[X+\lambda Y] = \Esper[X] + \lambda \Esper[Y]\]
 où $\lambda \in \setR$.
 
 \item Condition suffisante d'intégrabilité: soit $Y$ une variable aléatoire à densité et intégrable. Si $X$ est à densité et vérifie $\Proba(\abs{X}\leq Y)=1$ alors $X$ est intégrable.
 
 \item Croissance: Si $X$ et $Y$ sont à densité et intégrables, \[\Proba(X\leq1) = 1 \;\implique\; \Esper[X] \leq \Esper[Y]\]
\end{itemize}

\chapter{Vecteurs aléatoires à densité}

\section{Définition}

On dit que le vecteur aléatoire $X = (\Xoned): \Omega \mapsto \setR^d$ possède la densité $p: \setR^d \mapsto \setRp$ pour tout $o$ un ouvert de $\setR^d$.

\begin{align*}
 \Proba(X\in o) &= \int_o p(x) \dif x \\
 &= \int_{\setR^d} I_0(x) p(x) \dif x \\
 &= \int_{\setR^d} I_0(\xoned) p(\xoned) \dif \xoned
\end{align*}

\paragraph{Théorème}
Le vecteur aléatoire $X:\Omega \mapsto \setR^d$ possède la densité $p:\setR^d \mapsto \setRp$ si et seulement si pour toute fonction bornée $f:\setR^d \mapsto \setR$ on a que: \[\Esper[f(x)]=\int_{\setR^d}f(x)p(x) \dif x\]

\paragraph{Remarque}
Soient $X:\Omega \mapsto \setR^d$ de densité $p$ et $f:\setR^d \mapsto \setR$.

Si $\int_{\setR^d} \lvert f(x) \rvert p(x) \dif x < \infty$ alors $f(x)$ est dite \emph{intégrable} et:
\[ \Esper[f(x)] = \int_{\setR^d} f(x) p(x) \dif x \]

\section{Densité marginale}

\paragraph{Proposition}
Soit $X$ une variable aléatoire qui possède une densité, alors tout sous-vecteur $Y$ de $X$ possède une densité marginale obtenue en intégrant la densité de $X$ sur les composantes qui ne figurent pas dans $Y$.

\paragraph{Démonstration}
Soit $X=(\Xoned):\Omega \mapsto \setR^d$ un vecteur aléatoire à densité notée $p$, on considère $k<d$.

Le vecteur $(\Xoned)$ est-il à densité?
\[\forall O_k \subset \setR^k\]
\[\Proba((\Xoned) \in O_k) = \int_{\setR^k} I_{O_k}(\xonek) q(\xonek) \dif \xonek\]

\[\forall O_d \text{ de }\setR^d\]
\[\Proba((\Xoned) \in O_d) = \int_{\setR^d} I_{O_d}(x_1\dots x_k, x_{k+1}\dots x_d) q(\xoned) \dif \xoned\]

\section{Changement de variable}
On se pose la question suivante:
\begin{cquote}{}
 Si $X$ est une variable aléatoire qui possède une densité de valeurs dans $\setR^d$ et $f$ est la fonction $f:\setR^d \mapsto \setR^d$, est-ce que $Y=f(X)$ est à densité?
\end{cquote}

\paragraph{Proposition}
Soit $X:\Omega \mapsto \setR^d$ est un vecteur aléatoire qui possède la densité $p$ portée par un ouvert $o$ de $\setR^d$ (c'est-à-dire que $\int_o p(x) \dif x = 1$), et $\varphi$ une bijection de $o$ sur $G'$ de classe C', ainsi que son inverse $\varphi^{-1}$; alors $Y=f(X)$ possède la densité:
\[q(y) = I_{G'}(y)\;p(\varphi^{-1}(y)) \; \lvert \jac \varphi^{-1}(y) \rvert \]
où $\jac\varphi^{-1} (y)$ est le Jacobin de l'inverse de $\varphi$; et vaut:
\[ \jac \varphi^{-1}(y) = \det\left(\frac{\partial(\varphi^{-1}_i)}{\partial y_j}(y)\right)_{1\leq i\leq d, 1\leq j \leq d} \]

\section{Indépendance}

\paragraph{Définition}
Les vecteurs aléatoires $X_1:\Omega \mapsto \setR^{d_1}\dots X_n:\Omega \mapsto \setR^{d_n}$ qui possèdent les densités $p_1:\setR^{d_1}\mapsto \setR\dots p_n:\setR^{d_n} \mapsto \setR$, sont dites indépendantes si $X=(\Xonen)$ possède la densité $p_1(x_1) p_2(x_2) \dots p_n (x_n)$ telle que:

\begin{align*}
 X_1 &= \begin{pmatrix}X_{1,1}\\\vdots\\X_{1,d_1}\end{pmatrix} \qquad p_1(x_1) = p_1(x_{1,n},\dots x_{1,d_1})\\
 & \vdots \\
 X_n &= \begin{pmatrix}X_{n,1}\\\vdots\\X_{n,d_n}\end{pmatrix} \qquad p_n(x_n) = p_n(x_{n,1},\dots x_{n,d_n})
\end{align*}

\paragraph{Proposition}
Soient $X_1:\Omega \mapsto \setR^{d_1}, \dots X_n:\Omega \mapsto \setR^{d_n}$, $n$ variables aléatoires indépendantes, alors pour toute fonction $f_1:\setR^{d_1}\mapsto \setR, \dots f_n:\setR^{d_n}\mapsto \setR$ telle que $f_i(X_i)$ est intégrable:
\[\Esper[f_1(X_1)\: f_2(X_2) \dots f_n(X_n)] = \Esper[f_1(X_1)]\: \Esper[f_2(X_2)] \dots \Esper[f_n(X_n)]\]

\section{Covariance}
On rappelle que si $Y$ et $Z$ sont des variables de carré intégrable, alors le produit $YZ$ est intégrable car $\abs{YZ}\leq\frac{Y^2 + Z^2}{2}$.

\paragraph{Définition}
Soient $Y$ et $Z$ deux variables aléatoires réelles de carré intégrable. On appelle covariance de $Y$ et $Z$ le nombre noté $\Cov(Y,Z)$:
\[ \Cov(Y,Z)=\Esper\left[(Y-\Esper[Y])(Z-\Esper[Z])\right] \]

Soit $X=\Xoned$ un vecteur aléatoire à valeurs dans $\setR^d$ dont les composantes sont de carré intégrable. On appelle matrice de covariance du vecteur $X$ la matrice $K^X$ de dimension $d\times d$ où:
\[ K_{i,j}^X = \Esper[(X_i - \Esper[X_i]) (X_j-\Esper[X_j])] \]

\paragraph{Propriétés}
\begin{itemize}
 \item $\Cov(Y,Y)=\Var(Y)$ si $Y$ est à valeurs réelles.
 \item Les éléments sur la diagonale de $K^X$ sont les variances des composantes de $X$.
 \item $\Cov(Y,Z) = \Esper[YZ] - \Esper[Y]\;\Esper[Z]$
 \item Si $Y$ et $Z$ sont indépendantes, alors $\Cov(Y,Z)=0$.
 \item La matrice $X^k$ est symétrique et positive:
 \[ \transposee{(K^X)} = K^X \text{ et } \transposee{\!u} K^X u \geq 0 \] où $u\in\setR^d$.
\end{itemize}

\paragraph{Corrolaire}
Si $\Xoned$ sont des variables aléatoires réelles de carré intégrable et indépendantes;
\[ \Var\left(\sum_{i=1}^d X_i\right) = \sum_{i=1}^d \Var(X_i) \]

\chapter{Convergence et théorèmes limites}

Dans ce chapitre, on va introduire différents modes de convergence pour une suite de variables aléatoires. Cela va nous permettre d'établir deux résultats:
\begin{itemize}
 \item La loi forte des grands nombres, qui permet d'assurer la convergence de la moyenne empirique $\frac 1 n \sum_{i=1}^n X_i$ d'une suite IID $(X_i)_{i\geq1}$ de variable aléatoire vers $\Esper[X_i]$.
 
 \item Le théorème limite centrale, qui indique à quelle vitesse a lieu la convergence.
\end{itemize}

\section{Convergence}

\paragraph{Définition}
Pour $n\to+\infty$, on dit qu'une suite $(X_n)_{n\geq1}$ de variables aléatoires à valeurs dans $\setR^d$ converge vers la variable aléatoire $X$ à valeurs dans $\setR^d$:
\begin{itemize}
 \item Presque sûrement si $\Proba(\omega\in\Omega, \lim_{n\to\infty} X_n(\omega) = X(\omega)) = 1$.
 
 \item En probabilité: $\forall \varepsilon > 0$:
 \[\lim_{n\to\infty} \Proba(\omega\in\Omega, \abs{X_n(u) - X(u)} \geq \varepsilon) = 0\]
 
 \item Dans $L^1$: si $X$ et $X_n$ pour $n\geq1$ sont intégrables:
 \[ \lim_{n\to\infty} \Esper[\abs{X_n - X}] = 0 \]
 
 \item Dans $L^2$: si $X$ et $X_n$ pour $n\geq 1$ sont de carré intégrable:
 \[ \lim_{n\to\infty} \Esper\left[(X_n-X)^2\right] = 0\]
\end{itemize}

\paragraph{Remarque}
Soit $(X_n)_{n\geq1}$ une suite de variables aléatoires qui converge vers $X$ dans $L^1$; alors $\lim_{n\to\infty} \Esper[X_n]=\Esper[X]$, et \[X_n-X\leq\abs{X_n-X} \implique \lim_{n\to\infty} \Esper[X_n]\leq\Esper[X]\] et par symétrie on obtient l'autre inégalité.

\paragraph{Théorème de convergence dominée}
Soit $(X_n)_{n\geq1}$ une suite de variables aléatoires qui converge presque sûrement vers $X$. On suppose de plus que la suite $(X_n)_{n\geq1}$ est dominée au sens où où il existe une variable aléatoire $Y$ intégrable telle que $\Proba(\abs{X_n}\leq Y) = 1$ pour $n\geq1$; alors $X$ est intégrable et $(X_n)_{n\geq1}$ converge dans $L^1$ vers $X$ et en particulier, \[\lim_{n\to\infty} \Esper[X_n]=\Esper[X]\]

\paragraph{Proposition: inégalités}
\begin{itemize}
 \item Inégalité de Markov: si $\Esper[\abs{X}]<\infty$ alors $\forall a > 0$:
 \[\Proba(\abs{X}\geq a) \leq \frac{\Esper[\abs{X}]}a\]
 
 \item Inégalité de Bienaymé Tchebychev: si $\Esper[\abs{X^2}]<\infty$ alors $\forall a > 0$:
 \[\Proba(\abs{X}\geq a) \leq \frac{\Esper[X^2]}{a^2}\]
 
 \item Inégalité de Cauchy-Schwartz: si $X$ et $Y$ sont de carré intégrable alors \[\abs{\Esper[XY]}\leq\sqrt{\Esper[X^2]} \sqrt{\Esper[Y^2]}\]
\end{itemize}

\paragraph{Proposition}
\begin{itemize}
 \item La convergence dans $L^2$ implique la convergence dans $L^1$, implique la convergence en probabilité.
 
 \item La convergence presque sûre implique la convergence en probabilité.
 
 \item Si $(X_n)_{n\geq1}$ converge dans $L^2$ vers $X$ alors: \[\lim_{n\to\infty} \Esper[X_n] = \Esper[X]\] \[\lim_{n\to\infty}\Esper[X_n^2] = \Esper[X]\] \[\lim_{n\to\infty} \Var(X_n) = \Var(X)\]
\end{itemize}

\section{Loi des grands nombres}

\subsection{Loi faible des grands nombres}

\paragraph{Proposition}
La moyenne empirique $\overline X_n = \frac 1 n \sum_{k=1}^n X_k$ d'une suite $(X_k)_{k \geq 1}$ de variable aléatoire IID et de carré intégrable converge dans $L^2$ vers $\Esper[X_i]$, $\forall j \; \Esper[X_j] = \Esper[X_i]$.

\subsection{Loi forte des grands nombres}

\paragraph{Proposition}
La moyenne empirique d'une suite de variables aléatoires à valeurs réelles IID intégrables, converge presque sûrement et dans $L^1$ vers $\Esper[X_i]$.

\paragraph{Démonstration d'un cas particulier}
On suppose $\Esper[X_1^4]<\infty$ et on va montrer la convergence presque sûre vers $\Esper[X_i]$.
\begin{align*}
 \Esper\left[(\overline X_n - \Esper[X_1])^4\right] &= \Esper\left[\left(\frac 1 n \sum_{k=1}^n X_k - \frac 1 n \sum_{k=1}^n \Esper[X_k]\right)^4\right] \\
 &= \frac 1 {n^4}\;\Esper\left[\left(\sum_{k=1}^n X_k - \Esper[X_k]\right)^4\right] \\
 &= \frac 1 {n^4} \sum_{i_1 = 1}^n \sum_{i_2 = 1}^n \sum_{i_3 = 1}^n \sum_{i_4 = 1}^n \Esper[(X_{i_1}-\Esper[X_{i_1}])]\; \Esper[(X_{i_2}-\Esper[X_{i_2}])] \\ &\qquad\qquad\qquad\qquad\qquad\, \Esper[(X_{i_3}-\Esper[X_{i_3}])]\; \Esper[(X_{i_4}-\Esper[X_{i_4}])]
\end{align*}

On remarque que $\Esper[X_{i_1} - \Esper[X_{i_1}]] = 0$. On en déduit que les valeurs sont égales par paires:
\begin{align*}
 \Esper\left[(X_{i_1} - \Esper[X_{i_1}])^2\right]\;\Esper\left[(X_{i_2} - \Esper[X_{i_2}])^2\right] &= \Var(X_{i_1}) \Var(X_{i_2}) \\
 &= \Var(X_1)^2
\end{align*}
car les deux sont identiquement distribuées.

On peut donc reprendre l'égalité précédente;
\begin{align*}
 \Esper\left[(\overline X_n - \Esper[X_1])^4\right] &= \frac 1 {n^4} \left[n \Esper\left[(X_1 - \Esper[X_1])^4\right] + 3 n (n-1) \Var(X_1)^2\right]
\end{align*}
avec \[ \Esper\left[(\overline X_n - \Esper[X_i])^4\right] = \frac 1 {n^3}\; \Esper\left[(X_i - \Esper[X_1])^4\right] + 3 \frac{n-1}{n^3} \Var(X_i)^2 \]

On observe:
\begin{align*}
 \sum_{n=1}^\infty \Esper\left[(\overline X_n - \Esper[X_i])^4\right] < \infty &\implique \Esper\left[\sum_{n=1}^\infty (\overline X_n - \Esper[X_i])^4\right] < \infty \\
\end{align*}

On en déduit que:
\begin{align*}
 &\Proba(\omega \in \Omega: \sum_{k=1}^\infty (\overline X_n(\omega) - \Esper[X_1])^4 < \infty) = 1 \\
 \implique\;& \Proba(\omega \in \Omega: \lim_{n\to\infty} (\overline X_n(\omega) - \Esper[X_1])^4 < \infty = 0)\\
 \implique\;& \Proba(\omega \in \Omega: \lim_{n\to\infty} \overline X_n(\omega) = \Esper[X_1]) = 1
\end{align*}

\section{Fonction caractéristique et convergence en loi}

Il est nécessaire d'introduire la convergence en loi pour étudier la vitesse et convergence dans la loi forte (faible) des grands nombres.

\subsection{Fonction caractéristique}

\paragraph{Définition}
Soit $X$ un vecteur aléatoire $X:\Omega\mapsto\setR^d$. On appelle \emph{fonction caractéristique} de $X$ (notée $\Phi_X$) la fonction définie sur $\setR^d$:
\[ \Phi_X (u) = \Esper[e^{i\, \transposee{\!u}\, x}] \]
$\forall u \in \setR^d$ et $\transposee{\!u}\in\setR$.

\paragraph{Application aux lois connues}
\begin{itemize}
 \item Loi de Bernouilli de paramètre $p\in[0,1]$:
 \[ \Phi_X(u) = 1-p+pe^{iu} \] où $u\in\setR$.
 
 \item Loi binomiale de paramètres $n\in\setNs$ et $p\in[0,1]$:
 \[ \Phi_X(u) = (1-p + pe^{iu})^n \] où $u\in\setR$.
 
 \item Loi géométrique de paramètre $p\in]0,1]$:
 \[ \Phi_X(u) = \frac{pe^{iu}}{1-(1-p)e^{iu}} \] où $u\in\setR$.
 
 \item Loi uniforme sur $[a,b]$:
 \[ \Phi_X(u) = \frac{\sin((b-a)\frac u 2)}{(b-a)\frac u 2} e^{iu \frac{a+n}2} \] où $u\in\setR$.
 
 \item Loi exponentielle de paramètre $\lambda>0$:
 \[ \Phi_X(u) = \frac \lambda {\lambda^{-iu}} \] où $u\in\setR$.
 
 \item Loi gaussienne $\normale(\mu, \sigma^2)$:
 \[ \Phi_X(u) = \frac{\sigma^2 u^2}2 e^{iu\mu} \] où $u\in\setR$.
\end{itemize}

\section{Le théorème de limite centrale}

\paragraph{Théorème}
Soit $(X_j)_{j\geq1}$ ne ste de variables aléatoires réelles IID. telles que $\Esper[X_i^2]<+\infty$ et $\sigma=\sqrt{\Var(X_i)} > 0$. On note la moyenne empirique: \[\overline X_n = \frac1n \sum_{k=1}^n X_k\]

Alors on a:
\[\frac{\sqrt n}{\overline V} \left(\overline X_n - \Esper[X_i]\; \underset{n\to\infty}{\overset{\mathcal{L}}\to} \;Y \leadsto \normale(0, 1)\right)\]

\subsection{Intervalle de confiance et méthode de Monte Carlo}

\paragraph{Énoncé}
Le principe de la méthode de Monte Carlo est d'évaluer numériquement l'espérance d'une variable aléatoire réelle $X$ intégrable.

On génère une réalisation d'une suite de variables aléatoires $(X_j)_{j\geq1}$ IID. de même loi que $X$.

L'idée est d'approcher $\Esper[X]$ par $\overline X_n = \frac1n \sum_{k=1}^n X_k$. La loi forte des grands nombres assume la convergence de cette approximation.

Si on donne $f(x)=I_{\abs x \leq a}$ la fonction continue en dehors de $a$ et $-a$, qui est bornée. On peut alors dire que \[\lim_{n\to\infty} \Esper\left[f\left(\frac{\sqrt n}{\overline V} (\overline X_n - \Esper[X])\right)\right] = \Esper[f(Y)]\]
où $Y\leadsto\normale(0,1)$.

\paragraph{Explication rapide}
Pour $n$ grand on a que; \[\Proba\left[\overline X_n - \frac{a\sigma}{\sqrt n} \leq \Esper[X] \leq \overline X_n + \frac{a\sigma}{\sqrt n}\right] \approx \int_{-a}^a \frac1{\sqrt{2\pi}} e^{-\frac{y^2}2} \dif y\]

Pour un intervalle de confiance à 95\%, on prend $a=1.96$; et pour un intervalle à 99\%, on prend $a=2.58$.

Cela veut dire que l'on a confiance à 99\% que l'espérance $\Esper[X]$ soit comprise dans l'intervalle: \[\left[ \overline X_n - \frac{2.58\sigma}{\sqrt n};\; \overline X_n + \frac{2.58\sigma}{\sqrt n} \right]\]

Une fois que $a$ est fixé, il reste à choisir $n$ pour s'assurer de la précision, sachant que la taille de l'intervalle de confiance vaut $\frac{2\sigma a}{\sqrt n}$.

On approxime $\sigma$ par \[\overline\sigma_n = \sqrt{\frac1n \sum_{k=1}^n X_k^2 - (\overline X_n)^2}\]

On choisit alors pour intervalle de confiance:
\[I_n = \left[ \overline X_n - \frac{a\sigma_n}{\sqrt n};\; \overline X_n + \frac{a\sigma_n}{\sqrt n} \right]\]

\chapter{Remise à niveau}

\section{Fonctions et suites génératrices}

\paragraph{Définition}
On a une suite $(a_n)_{n=0}^\infty$, on peut considérer la fonction:
\[f(x) = \sum_{n=0}^\infty a_n x^n\]
Attention, cette fonction n'est pas forcément définie en 0. $f$ est toujours définie sur:
\begin{itemize}
 \item $\lbrace 0 \rbrace$,
 \item $\setR$,
 \item $] -r, r [$ pour un certain $r$, appelé ``rayon de convergence'',
 \item $[ -r, r ]$ (idem).
\end{itemize}
On a souvent $r=1$.

\paragraph{Propriétés}
Supposons qu'on ai montré que si $r>0$, alors $f(x)=\sum_{n=0}^\infty a_n x^n$ sur $]-r,r[$. On a alors:
\begin{itemize}
 \item $f$ est infiniment dérivable,
 \item $\frac{f^{(n)} (x)}{n!} = a_n$ donc $f^{(n)}(x) = a_n n!$,
 \item \[f'(x) = \sum_{n=1}^\infty n a_n x^{n-1}\]
 \item \[F(x) = \sum_{n=0}^\infty \frac{a_n}{a+1} x^{n+1}\]
\end{itemize}

\paragraph{Appliquée aux probabilités}
En probabilités, la fonction génératrice est définie par:
\begin{itemize}
 \item $X$ à valeurs dans $\setN$,
 \item \[g_x (t) = \Esper[t^X] = \sum_{n=0}^\infty \Proba(X=n) t^n\] (où $\Proba(X=n)$ représente $a_n$).
\end{itemize}
On en déduit que le rayon de convergence vaut au moins 1, car pour tout $t=1$: \[\sum_{n=0}^\infty \Proba(X=n)=1\] donc on sait tout de suite que $g_x (t)$ est définie sur $[-1,1]$ au minimum.

\paragraph{Caractérisation de la loi}
La fonction génératrice caractérise la loi:
Si $\forall t \in ]-1,1[$ on a \[g_x(t) = g_y(t)\] alors $X$ et $Y$ ont la même loi ($X \laweq Y$).

À noter que si $g'_x(1)$ et $\Esper[X]$ existent, on a: \[g'_x(1) = \Esper[X]\]

\end{document}
