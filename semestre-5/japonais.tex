\documentclass[a4paper,10pt,french,openany]{memoir}
\usepackage[utf8]{inputenc}
\usepackage{babel}

\let\printglossary\undefined
\usepackage{luatexja}
\usepackage{luatexja-fontspec}

\usepackage{clovisai}

%opening
\title{Japonais}
\author{??\\Notes prises par Ivan Canet}%TODO

\begin{document}

\maketitle
\tableofcontents

\chapter{Hiraganas}
\section{Simples}

Les hiraganas (ひらがな) représentent des syllabes (\autoref{tab:hira-simple}).

\begin{table}[htbp]
 \centering
 \begin{tabular}{llllll}
  あ a  & い i  & う u  & え e  & お o  &\\
  か ka & き ki & く ku & け ke & こ ko &\\
  さ sa & し shi& す su & せ se & そ so &\\
  た ta & ち chi& つ tsu& て te & と to &\\
  な na & に ni & ぬ nu & ね ne & の no &\\
  は ha & ひ hi & ふ fu & へ he & ほ ho &\\
  ま ma & み mi & む mu & め me & も mo &\\
  や ya &      & ゆ yu &      & よ yo &\\
  ら ra & り ri & る ru & れ re & ろ ro &\\
  わ wa &      &      &      & を wo & ん n\\
 \end{tabular}
 \caption{Tableaux des hiraganas}
 \label{tab:hira-simple}
\end{table}

\section{Dakuten \& Handakuten}

Des sons plus compliqués peuvent être formés en combinant les hiraganas simples avec un dakuten ('') ou un handakuten (°) (\autoref{tab:dakuten-handakuten}).

\begin{table}[htbp]
 \centering
 \begin{tabular}{r|lllll}
  /k/'': /g/ & が ga & ぎ gi & ぐ gu & げ ge & ご go \\
  /s/'': /z/ & ざ za & じ ji & ず zu & ぜ ze & ぞ zo \\
  /t/'': /d/ & だ da & ぢ ji & ず zu & で de & ど do \\
  /h/'': /b/ & ば ba & び bi & ぶ bu & べ be & ぼ bo \\
  /h/°: /p/  & ぱ pa & ぴ pi & ぷ pu & ぺ pe & ぽ po \\
 \end{tabular}
 \caption{Effets du dakuten ('') et du handakuten (°)}
 \label{tab:dakuten-handakuten}
\end{table}

\section{Liaisons sur le son /i/}

Les sons du type /nyu/ sont obtenus en combinant un hiragana se terminant par le son /i/ (colonne du /i/) avec un autre commençant par le son /i/ (ligne du /y/). Attention, certains perdent le son /i/ (par exemple, /shi/ + /ya/ donne /sha/ et non /shya/).

Dans ce cas, le deuxième hiragana est écrit en petit (pour faire la différence entre きゃ /kya/ et きや /kiya/). Les résultats sont présentés en \autoref{tab:hira-liasons-i}.

\begin{table}[htbp]
 \centering
 \begin{tabular}{lll}
  きゃ kya& きゅ kyu& きょ kyo\\
  ぎゃ gya& ぎゅ gyu& ぎょ gyo\\
  しゃ sha& しゅ shu& しょ sho\\
  じゃ ja & じゅ ju & じょ jo \\
  ちゃ cha& ちゅ chu& ちょ cho\\
  にゃ nya& にゅ nyu& にょ nyo\\
  ひゃ hya& ひゅ hyu& ひょ hyo\\
  びゃ bya& びゅ byu& びょ byo\\
  ぴゃ pya& ぴゅ pyu& ぴょ pyo\\
  みゃ mya& みゅ myu& みょ myo\\
  りゃ rya& りゅ ryu& りょ ryo\\
 \end{tabular}
 \caption{Liaisons sur le son /i/}
 \label{tab:hira-liasons-i}
\end{table}

\section{Consonnes et voyelles allongées}

En écrivant un petit つ (tsu), on peut signifier qu'il faut faire une pause dans le mot. En rōmaji, on double la consonne suivante. Le \autoref{tab:petit-tsu} donne des exemples.

\begin{table}[htbp]
 \centering
 \begin{tabular}{ll}
  たつと tatsuto & たっと tatto \\
  きて kite (aller) & きって kitte (timbre)
 \end{tabular}
 \caption{Exemples d'utilisation du petit /tsu/}
 \label{tab:petit-tsu}
\end{table}

À noter que quand on souhaite doubler le son /n/ seul, on n'écrit pas de petit /tsu/. Voir le tableau \autoref{tab:petit-tsu-n} pour des exemples.

\begin{table}[htbp]
 \centering
 \begin{tabular}{l}
  さんねん sannen\\
  あんない annai
 \end{tabular}
 \caption{On n'utilise pas de petit /tsu/ pour le son /n/}
 \label{tab:petit-tsu-n}
\end{table}

Pour écrire les voyelles longues, on écrit une voyelle supplémentaire. Par exemple, okaasan (mère) s'écrit おかあさん. En rōmaji, on a le choix entre doubler la voyelle ou écrire un accent plat pour montrer qu'elle est accentuée. D'autres exemples sont donnés dans le \autoref{tab:ex-doublage}. Attention, certains sons peuvent être obtenus de deux différentes manières, exemples dans le \autoref{tab:long-sounds}.

\begin{table}[htbp]
 \centering
 \begin{tabular}{rl}
  mère & おかあさん okāsan \\
  père & おとうさん otōsan\\
  grand-frère & おにいさん onīsan\\
  grande-sœur & おねえさん onēsan\\
  air & くうき kūki\\
  professeur\cdot{}e & せんせい sensē\\
  nager & すいえい suiē\\
  glace & こおり kōri\\
  grand & おおきい ōkī\\
 \end{tabular}
 \caption{Exemples d'appui sur la voyelle}
 \label{tab:ex-doublage}
\end{table}

\begin{table}[htbp]
 \centering
 \begin{tabular}{cll}
  ā & doubler le $a$ (おかあさん okāsan)\\
  ī & doubler le $i$ (おにいさん onīsan)\\
  ū & doubler le $u$ (くうき kūki)\\
  ē & doubler le $e$ & ou ajouter un $i$ (せんせい sensē)\\
  ō & doubler le $o$ (おおき ōki) & ou ajouter un $u$ (おうえん ōen)\\
 \end{tabular}
 \caption{Différentes manières de former les sons longs}
 \label{tab:long-sounds}
\end{table}

\chapter{Compter}
\section{Jusqu'à 100}

Les chiffres de base sont représentés dans le \autoref{tab:chiffres}.

\begin{table}[htbp]
 \centering
 \begin{tabular}{rll}
  0 & れい  rei\\
  1 & いち  ichi\\
  2 & に   ni\\
  3 & さん  san\\
  4 & よん  yon & し  shi\\
  5 & ご   go\\
  6 & ろく  roku\\
  7 & なな  nana & しち shichi\\
  8 & はち  hachi\\
  9 & きゅう kyū & く  ku\\
 10 & じゅう jū
 \end{tabular}
 \caption{Nombres de 0 à 10}
 \label{tab:chiffres}
\end{table}

Pour former les nombres jusqu'à 99, on écrit le numéro de dizaine, 10, puis les unités. Par exemple, 25 s'écrit 2-10-5, donc にじゅうご (nijūgo). Le \autoref{tab:dizaines} présente des exemples.

\begin{table}[htbp]
 \centering
 \begin{tabular}{cll}
  11 & じゅういち      &jū ichi\\
  22 & にじゅうに      &nijū ni\\
  33 & さんじゅうさん   &sanjū san\\
  44 & よんじゅうよん   &yonjū yon\\
  55 & ごじゅうご      &gojū go\\
  66 & ろくじゅうろく   &rokujū roku\\
  77 & ななじゅうなな   &nanajū nana\\
  88 & はちじゅうはち   &hachijū hachi\\
  99 & きゅうじゅうきゅう&kyūjū kyū\\
 \end{tabular}
 \caption{Exemples de nombres de 11 à 99}
 \label{tab:dizaines}
\end{table}

\section{Au-delà de 100}

Les centaines sont marquées avec ひゃく (hyaku). Attention à la prononciation pour 300, 600 et 800, marqués par un `×' dans le \autoref{tab:centaines}.

\begin{table}[htbp]
 \centering
 \begin{tabular}{clll}
  100 & ひゃく      &hyaku\\
  200 & にひゃく    &nihyaku\\
  300 & さんびゃく   &sanbyaku&×\\
  400 & よんひゃく   &yonhyaku\\
  500 & ごひゃく    &gohyaku\\
  600 & ろっぴゃく   &roppyaku&×\\
  700 & ななひゃく   &nanahyaku\\
  800 & はっぴゃく   &happyaku&×\\
  900 & きゅうひゃく &kyūhyaku\\
 \end{tabular}
 \caption{Tableau des centaines}
 \label{tab:centaines}
\end{table}

Les milliers sont marqués avec せん (sen). Attention à la prononciation pour 3~000 et 8~000, marqués par un `×' dans le \autoref{tab:milliers}.

\begin{table}[htbp]
 \centering
 \begin{tabular}{clll}
  1~000 & せん       &sen\\
  2~000 & にせん      &nisen\\
  3~000 & さんぜん    &sanzen&×\\
  4~000 & よんせん    &yonsen\\
  5~000 & ごせん      &gosen\\
  6~000 & ろくせん    &rokusen\\
  7~000 & ななせん    &nanasen\\
  8~000 & はっせん    &hassen&×\\
  9~000 & きゃううせん &kyūsen\\
 \end{tabular}
 \caption{Tableau des milliers}
 \label{tab:milliers}
\end{table}

Pour écrire les nombres de 10~000 à 90~000, on utilise まん (man). Il n'y a pas d'exceptions.

\section{Compter de l'argent}

L'unité d'argent est le 円 (えん, en). Pour parler d'argent, on suffixe simple le nombre par le kanji 円, par exemple 1234~¥ s'écrit せんにひゃくさんじゅうよん円 (sen nihyaku sanjū yon en).

Pour demander combien coûte quelque chose, on utilise l'interrogatif いくら (ikura):
\begin{cquote}{}
 この本いくらですか (kono hon wa ikura desu ka, combien coûte ce livre\footnote{`Livre' s'écrit 本 (ほん, hon).} ?)
  
 にせん円です (nisen en desu, 2~000~¥)
\end{cquote}


\end{document}
