\documentclass[a4paper,10pt,french,openany]{memoir}
\usepackage[utf8]{inputenc}
\usepackage{babel}

\let\printglossary\undefined
\usepackage{luatexja}
\usepackage{luatexja-fontspec}
\usepackage{luatexja-ruby}

\usepackage{clovisai}

\newcommand{\term}[1]{\textcolor{blue}{#1}}

%opening
\title{Japonais}
\author{愛里せんせい\\Notes prises par Ivan Canet}

\addto{\captionsfrench}{\renewcommand{\abstractname}{Introduction}}

\begin{document}

\maketitle

\begin{abstract}
 Notes de cours prises lors des cours de Japonais dispensés à l'ENSEIRB-MATMECA, de 2019 à 2020.
 
 Ces notes s'organisent en plusieurs chapitres indépendants, sur les différentes parties de la langue.
 Au sein de chaque chapitre, on essaie d'avoir un avancement logique se basant sur les sections précédentes, mais qui peut aussi se baser sur des sections d'autres chapitres.
\end{abstract}


\tableofcontents

\chapter{Lecture}
\section{Hiraganas}

Le Japonais est écrit avec trois différents alphabets, les hiraganas, les katakanas, et les kanjis.

\subsection{Tableau des hiraganas sans combinaisons}

Les hiraganas (ひらがな) représentent des syllabes (\autoref{tab:hira-simple}).

\begin{table}[htbp]
 \centering
 \caption{Tableaux des hiraganas}
 \label{tab:hira-simple}
 \begin{tabular}{llllll}
  あ a  & い i  & う u  & え e  & お o  &\\
  か ka & き ki & く ku & け ke & こ ko &\\
  さ sa & し shi& す su & せ se & そ so &\\
  た ta & ち chi& つ tsu& て te & と to &\\
  な na & に ni & ぬ nu & ね ne & の no &\\
  は ha & ひ hi & ふ fu & へ he & ほ ho &\\
  ま ma & み mi & む mu & め me & も mo &\\
  や ya &      & ゆ yu &      & よ yo &\\
  ら ra & り ri & る ru & れ re & ろ ro &\\
  わ wa &      &      &      & を wo & ん n\\
 \end{tabular}
\end{table}

\subsection{Dakuten \& Handakuten}

Des sons plus compliqués peuvent être formés en combinant les hiraganas simples avec un dakuten ('') ou un handakuten (°) (\autoref{tab:dakuten-handakuten}).

\begin{table}[htbp]
 \centering
 \caption{Effets du dakuten ('') et du handakuten (°)}
 \label{tab:dakuten-handakuten}
 \begin{tabular}{r|lllll}
  /k/'': /g/ & が ga & ぎ gi & ぐ gu & げ ge & ご go \\
  /s/'': /z/ & ざ za & じ ji & ず zu & ぜ ze & ぞ zo \\
  /t/'': /d/ & だ da & ぢ ji & ず zu & で de & ど do \\
  /h/'': /b/ & ば ba & び bi & ぶ bu & べ be & ぼ bo \\
  /h/°: /p/  & ぱ pa & ぴ pi & ぷ pu & ぺ pe & ぽ po \\
 \end{tabular}
\end{table}

\subsection{Liaisons sur le son /i/}

Les sons du type /nyu/ sont obtenus en combinant un hiragana se terminant par le son /i/ (colonne du /i/) avec un autre commençant par le son /i/ (ligne du /y/). Attention, certains perdent le son /i/ (par exemple, /shi/ + /ya/ donne /sha/ et non /shya/).

Dans ce cas, le deuxième hiragana est écrit en petit (pour faire la différence entre きゃ /kya/ et きや /kiya/). Les résultats sont présentés en \autoref{tab:hira-liasons-i}.

\begin{table}[htbp]
 \centering
 \caption{Liaisons sur le son /i/}
 \label{tab:hira-liasons-i}
 \begin{tabular}{lll}
  きゃ kya& きゅ kyu& きょ kyo\\
  ぎゃ gya& ぎゅ gyu& ぎょ gyo\\
  しゃ sha& しゅ shu& しょ sho\\
  じゃ ja & じゅ ju & じょ jo \\
  ちゃ cha& ちゅ chu& ちょ cho\\
  にゃ nya& にゅ nyu& にょ nyo\\
  ひゃ hya& ひゅ hyu& ひょ hyo\\
  びゃ bya& びゅ byu& びょ byo\\
  ぴゃ pya& ぴゅ pyu& ぴょ pyo\\
  みゃ mya& みゅ myu& みょ myo\\
  りゃ rya& りゅ ryu& りょ ryo\\
 \end{tabular}
\end{table}

\subsection{Consonnes et voyelles allongées}

En écrivant un petit つ (tsu), on peut signifier qu'il faut faire une pause dans le mot. En rōmaji, on double la consonne suivante. Le \autoref{tab:petit-tsu} donne des exemples.

\begin{table}[htbp]
 \centering
 \caption{Exemples d'utilisation du petit /tsu/}
 \label{tab:petit-tsu}
 \begin{tabular}{ll}
  たつと tatsuto & たっと tatto \\
  きて kite (aller) & きって kitte (timbre)
 \end{tabular}
\end{table}

À noter que quand on souhaite doubler le son /n/ seul, on n'écrit pas de petit /tsu/. Voir le tableau \autoref{tab:petit-tsu-n} pour des exemples.

\begin{table}[htbp]
 \centering
 \caption{On n'utilise pas de petit /tsu/ pour le son /n/}
 \label{tab:petit-tsu-n}
 \begin{tabular}{l}
  さんねん sannen\\
  あんない annai
 \end{tabular}
\end{table}

Pour écrire les voyelles longues, on écrit une voyelle supplémentaire. Par exemple, okaasan (mère) s'écrit おかあさん. En rōmaji, on a le choix entre doubler la voyelle ou écrire un accent plat pour montrer qu'elle est accentuée. D'autres exemples sont donnés dans le \autoref{tab:ex-doublage}. Attention, certains sons peuvent être obtenus de deux différentes manières, exemples dans le \autoref{tab:long-sounds}.

\begin{table}[htp]
 \centering
 \caption{Exemples d'appui sur la voyelle}
 \label{tab:ex-doublage}
 \begin{tabular}{rl}
  mère & おかあさん okāsan \\
  père & おとうさん otōsan\\
  grand-frère & おにいさん onīsan\\
  grande-sœur & おねえさん onēsan\\
  air & くうき kūki\\
  professeur\cdot{}e & せんせい sensē\\
  nager & すいえい suiē\\
  glace & こおり kōri\\
  grand & おおきい ōkī\\
 \end{tabular}
\end{table}

\begin{table}[htp]
 \centering
 \caption{Différentes manières de former les sons longs}
 \label{tab:long-sounds}
 \begin{tabular}{cll}
  ā & doubler le $a$ (おかあさん okāsan)\\
  ī & doubler le $i$ (おにいさん onīsan)\\
  ū & doubler le $u$ (くうき kūki)\\
  ē & doubler le $e$ & ou ajouter un $i$ (せんせい sensē)\\
  ō & doubler le $o$ (おおき ōki) & ou ajouter un $u$ (おうえん ōen)\\
 \end{tabular}
\end{table}

\section{Compter}
\subsection{Jusqu'à 100}

Les chiffres de base sont représentés dans le \autoref{tab:chiffres}.

\begin{table}[htp]
 \centering
 \caption{Nombres de 0 à 10}
 \label{tab:chiffres}
 \begin{tabular}{rlll}
  0 & 零 & れい\\
  1 & 一 & いち\\
  2 & 二 & に\\
  3 & 三 & さん\\
  4 & 四 & よん & し\\
  5 & 五 & ご\\
  6 & 六 & ろく\\
  7 & 七 & なな & しち\\
  8 & 八 & はち\\
  9 & 九 & きゅう & く\\
 10 & 十 & じゅう
 \end{tabular}
\end{table}

Pour former les nombres jusqu'à 99, on écrit le numéro de dizaine, 10, puis les unités. Par exemple, 25 s'écrit 2-10-5, donc にじゅうご (nijūgo). Le \autoref{tab:dizaines} présente des exemples.

\begin{table}[htbp]
 \centering
 \caption{Exemples de nombres de 11 à 99}
 \label{tab:dizaines}
 \begin{tabular}{crll}
  11 &  十一 & \term{じゅう}いち      &jū ichi\\
  22 & 二十二 & に\term{じゅう}に      &nijū ni\\
  33 & 三十三 & さん\term{じゅう}さん   &sanjū san\\
  44 & 四十四 & よん\term{じゅう}よん   &yonjū yon\\
  55 & 五十五 & ご\term{じゅう}ご      &gojū go\\
  66 & 六十六 & ろく\term{じゅう}ろく   &rokujū roku\\
  77 & 七十七 & なな\term{じゅう}なな   &nanajū nana\\
  88 & 八十八 & はち\term{じゅう}はち   &hachijū hachi\\
  99 & 九十九 & きゅう\term{じゅう}きゅう&kyūjū kyū\\
 \end{tabular}
\end{table}

\subsection{Au-delà de 100}

\paragraph{100 -- 999}
Les centaines sont marquées avec \ruby 百{ひゃく}. Attention à la prononciation pour 300, 600 et 800, marqués par un `×' dans le \autoref{tab:centaines}.

\begin{table}[p]
 \centering
 \caption{Tableau des centaines}
 \label{tab:centaines}
 \begin{tabular}{crrll}
  100 & 百 & \term{ひゃく}      &hyaku\\
  200 & 二百 & に\term{ひゃく}    &nihyaku\\
  300 & 三百 & さん\term{びゃく}   &sanbyaku&×\\
  400 & 四百 & よん\term{ひゃく}   &yonhyaku\\
  500 & 五百 & ご\term{ひゃく}    &gohyaku\\
  600 & 六百 & ろっ\term{ぴゃく}   &roppyaku&×\\
  700 & 七百 & なな\term{ひゃく}   &nanahyaku\\
  800 & 八百 & はっ\term{ぴゃく}   &happyaku&×\\
  900 & 九百 & きゅう\term{ひゃく} &kyūhyaku\\
 \end{tabular}
\end{table}

\paragraph{1~000 -- 9~999}
Les milliers sont marqués avec \ruby 千{せん}. Attention à la prononciation pour 3~000 et 8~000, marqués par un `×' dans le \autoref{tab:milliers}.

\begin{table}[p]
 \centering
 \caption{Tableau des milliers}
 \label{tab:milliers}
 \begin{tabular}{crrll}
  1~000 & 千 & \term{せん} &sen\\
  2~000 & 二千 & に\term{せん}      &nisen\\
  3~000 & 三千 & さん\term{ぜん}    &sanzen&×\\
  4~000 & 四千 & よん\term{せん}    &yonsen\\
  5~000 & 五千 & ご\term{せん}      &gosen\\
  6~000 & 六千 & ろく\term{せん}    &rokusen\\
  7~000 & 七千 & なな\term{せん}    &nanasen\\
  8~000 & 八千 & は\term{っせん}    &hassen&×\\
  9~000 & 九千 & きゃう\term{せん} &kyūsen\\
 \end{tabular}
\end{table}

\paragraph{10~000 -- 99~999~999}
Pour écrire les nombres de 10~000, on utilise \ruby 万{まん}. Il n'y a pas d'exceptions, à part que l'on écrit \ruby{一万}{いちまん} (10~000) et pas juste 万.

Pour les rangs suivants, on va combiner les résultats précédents : prenons l'exemple de 57~613~495, le japonais se comporte comme si on groupait les chiffres par quatre: 5761~3495.
\begin{itemize}
 \item 5 unités,
 \item 9 dizaines,
 \item 4 centaines,
 \item 3 milliers;
 \item 1 dix-millier,
 \item 6 dizaines de dix-milliers,
 \item 7 centaines de dix-milliers,
 \item 5 milliers de dix-milliers.
\end{itemize}
En lisant la décomposition de bas en haut, on trouve 五千 七百 六十 一万 三千 四百 九十 五.

\subsection{Compteurs}\label{sec:compteurs}

Pour compter des objets, on va désigner une `unité', appelée `compteur'. Le compteur se place derrière le nombre pour exprimer à quoi il correspond.

\begin{itemize}
 \item 二十円 est 20 suivi de `en', c'est donc une somme d'argent,
 \item 一じ est 1 suivi de `ji', c'est donc un horaire.
\end{itemize}

Il existe un grand nombre de compteurs, nous allons ici en voir certains.

\paragraph{Argent}

L'unité d'argent est le \ruby 円{えん}. Pour parler d'argent, on suffixe simple le nombre par le kanji 円, par exemple 1234~¥ s'écrit 千二百三十四円.

Pour demander combien coûte quelque chose, on utilise l'interrogatif いくら (ikura):
\begin{cquote}{}
 この\ruby{本}{ほん}いくらですか (combien coûte ce livre ?)
  
 二千円 (2~000~¥)
\end{cquote}

\paragraph{Heures}

Les japonais lisent l'heure en format 12 heures: 0--12 heures du matin, puis 0--12 heures du soir. On lit l'heure en disant le nombre des heures, puis じ, puis le nombre de minutes. La prononciation change selon les heures, voir le \autoref{tab:heures}.

\begin{table}[htbp]
 \centering
 \caption{Tableau des heures}
 \label{tab:heures}
 \begin{tabular}{crll}
  1h & いち\term{じ} &ichiji\\
  2h & に\term{じ} &niji\\
  3h & さん\term{じ} &sanji\\
  4h & よ\term{じ} &yoji&×\\
  5h & ご\term{じ} &goji\\
  6h & ろく\term{じ} &rokuji\\
  7h & しち\term{じ} &shichiji\\
  8h & はち\term{じ} &hachiji\\
  9h & く\term{じ} &kuji\\
  10h& じゅう\term{じ} &jūji\\
  11h& じゅうい\term{じ}&jūiji\\
  12h& じゅうに\term{じ}&jūniji\\
  1h 30& いち\term{じ}&ichiji han
 \end{tabular}
\end{table}

Quand on veut donner une durée en minutes, on utilise le suffixe ぷん. Voir le \autoref{tab:minutes}.

\begin{table}[htbp]
 \centering
 \caption{Tableau des minutes}
 \label{tab:minutes}
 \begin{tabular}{crlrl}
  1 & いっ\term{ぷん} &ippun\\
  2 & に\term{ふん} &nifun\\
  3 & さん\term{ぷん} &sanpun\\
  4 & よん\term{ぷん} &yonpun\\
  5 & ご\term{ふん} &gofun\\
  6 & ろっ\term{ぷん} &roppun\\
  7 & なな\term{ふん} &nanafun\\
  8 & はっ\term{ぷん} &happun & はち\term{ふん} & hachifun\\
  9 & きゅう\term{ふん} &kyūfun\\
  10& じゅっ\term{ぷん} &juppun & じっ\term{ぷん} & jippun\\
  11& じゅういっ\term{ぷん} &jūippun\\
  12& じゅうに\term{ふん} &jūnifun\\
  13& じゅうさん\term{ぷん} &jūsanpun\\
  14& じゅうよん\term{ぷん} &jūyonpun\\
  15& じゅうご\term{ふん} &jūgofun\\
  16& じゅうろっ\term{ぷん} &jūroppun\\
  17& じゅうなな\term{ふん} &jūnanafun\\
  18& じゅうはっ\term{ぷん} &jūhappun & じゅうはち\term{ふん} & jūhachifun\\
  19& じゅうきゅう\term{ふん}&jūkyūfun\\
  20& にじゅっ\term{ぷん} &nijuppun & にじっ\term{ぷん} & nijippun\\
  30& さんじゅっ\term{ぷん} &sanjuppun & さんじっ\term{ぷん} & sanjippun\\
 \end{tabular}
\end{table}

\paragraph{Âge}

L'âge est déterminé en utilisant le suffixe さい (sai). Voir le \autoref{tab:age}.

\begin{table}[htbp]
 \centering
 \caption{Tableau des âges}
 \label{tab:age}
 \begin{tabular}{rrlll}
  1 an  & い\term{っさい} &issai&×\\
  2 ans & に\term{さい} &nisai\\
  3 ans & さん\term{さい} &sansai\\
  4 ans & よん\term{さい} &yonsai\\
  5 ans & ご\term{さい} &gosai\\
  6 ans & ろく\term{さい} &rokusai\\
  7 ans & なな\term{さい} &nanasai\\
  8 ans & は\term{っさい} &hassai&×\\
  9 ans & きゅう\term{さい} &kyūsai\\
  10 ans& じゅ\term{っさい} &jussai\\
  11 ans& じゅうい\term{っさい}&jūissai & じ\term{っさい} & jissai\\
  20 ans& はたち &hatachi
 \end{tabular}
\end{table}

\subsection{Autres nombres}

\paragraph{Numéro de téléphone}

Pour lire un numéro de téléphone, on lit simplement les chiffres indépendamment (pas de groupement).

Il est aussi possible d'insérer の entre le code de zone (les 3 premiers chiffres) et le reste du numéro.

Le numéro de téléphone se dit でんわばんごう.

\chapter{Grammaire}
\section{Forme longue}

\subsection{Particules}

En Japonais, une phrase est composée d'un ensemble de blocs liés par des particules (écrites en hiraganas). Une phrase se termine soit pas un verbe, soit par です.

Le Japonais est une langue très contextuelle: en temps normal, on ne précise pas les informations qui peuvent être devinées grâce au contexte.

\paragraph{Groupes nominaux}
Les phrases sont organisées autour de groupe nominaux. Un groupe nominal est composé d'un ou plusieurs noms à la suite.

Par exemple, だいがく (université) est un groupe nominal, comme さくらだいがく (Université Sakura).

\begin{table}[htbp]
 \centering
 \caption{Exemples d'utilisation de です}
 \label{tab:desu}
 \begin{tabular}{ll}
  がくせい\term{です} & (Je) suis étudiant \\
  日本ご\term{です} & (Ma filière) est le japonais\\
  にじゅういちさい\term{です} & (Mon âge) est 21 ans\\
 \end{tabular}
\end{table}

On retrouve des exemples de groupes nominaux suivis de です dans le \autoref{tab:desu}.
On remarque que les informations entre parenthèses dans le \autoref{tab:desu} ne sont pas comprises dans la phrase: on les connaît grâce au contexte. がくせいです peut vouloir dire ``je suis étudiant'' mais aussi ``tu es étudiant'', ``elle est étudiante'', ``nous sommes étudiants'', ``vous êtes étudiants'', etc. Il est donc très important de faire attention au contexte.
De la même manière, じゅうにさいです peut vouloir dire ``j'ai 20 ans'' mais aussi ``je fais de la musique depuis 20 ans'', etc, selon le contexte.

\paragraph{は}
La particule は (prononcée /wa/) permet de définir qui fait l'action. On l'utilise quand on souhaite expliciter le sujet. Par exemple, on peut écrire わたし\term{は}マリナです (わたし: `je', は: particule du sujet, マリナ: prénom, です: terminaison).

Syntaxe: ``… nom \term{は} …''.

\paragraph{を et が}
Les particules を (prononcée /o/) et が définissent qui subit l'action. Par exemple, たこやき\term{を}たべます (たこやき: boulettes de poulpe, を: particule de l'objet, たべます: verbe `manger' conjugué).

Syntaxe: ``… nom \term{を} …'', ou ``… nom \term{が} …''.

En règle générale, les verbes d'actions sont précédés de を (eg. する), alors que les verbes d'états sont précédés de が (eg. ある), mais il existe aussi de nombreuses exceptions.

\paragraph{の et な}
La particule の permet d'ajouter un niveau de précision à un groupe nominal. Elle se place entre deux groupes nominaux, et exprime que le premier précise le second. Par exemple, dans la phrase さくらだいがく\term{の}がくせえ (さくらだいがく: l'université Sakura, の: particule de précision, がくせい: étudiant) permet de définir qu'on ne parle pas de n'importe quel étudiant, mais spécifiquement d'un étudiant de l'université Sakura. Plus d'exemples sont présentés dans le \autoref{tab:no}.

\begin{table}[htbp]
 \centering
 \caption{Exemples d'utilisation de の}
 \label{tab:no}
 \begin{tabular}{ll}
  たけしさあん\term{の}でんわばんごう & Le numéro de téléphone de Takeshi \\
  だいがく\term{の}せんせい & Professeur d'université \\
  日本ご\term{の}がくせい & Étudiant de la langue japonaise \\
  日本\term{の}だいがく & Une université japonaise
 \end{tabular}
\end{table}

Syntaxe: ``…~nom\textsubscript{précision} \term{の} nom\textsubscript{principal}~…''.

Pour avoir exactement la même relation que の, mais avec un adjectif \cf{sec:adjectifs}, on va utiliser:
\begin{itemize}
 \item Rien pour les い-adjectifs: ``…~adjectif nom~…''
 \item な pour les な-adjectifs: ``…~adjectif \term{な} nom~…''
\end{itemize}

\paragraph{で}
La particule で représente la position géographique. On l'utilise pour préciser l'endroit où la phrase a lieu, par exemple 東京\term{で}たべました (東京: Tōkyō, で: position géographique, たべました: verbe `manger' au passé) signifie ``j'ai mangé à Tokyo''.

Syntaxe: ``…~nom\textsubscript{endroit} \term{で}~…''.

\paragraph{に et へ}
Les particules に et へ (e) peuvent être utilisées pour donner le but d'une action. Par exemple, la phrase 日本\term{へ}ようこそ (日本: le Japon, へ: le but, ようこそ: `bienvenue') signifie ``Bienvenue au Japon''.

La particule に peut aussi être utilisée pour définir à quel moment une action à lieu (la position temporelle). On n'utilise pas に pour désigner des dates relatives (derrière ``aujourd'hui'', ``demain''\dots), les intervalles (``tous les jours''), ainsi que derrière le mot いつ (``quand''). On évite に derrière des périodes de la journée (``le matin'', ``le soir'') mais ce n'est pas interdit.

Syntaxe: ``…~nom \term{へ}~…'' ou ``…~nom \term{に}~…''.

\paragraph{も}
La particule も permet d'exprimer la similarité entre deux phrases: ``Nolan a mangé, David aussi.''
Cette particule précède un verbe dans une seconde phrase: ``nom\textsubscript{1} は/が/を verbe\textsubscript{1}, nom\textsubscript{2} \term{も} verbe\textsubscript{1}.''

\paragraph{と}
La particule と représente le `et' : リス\term{と}おにぎりをたべました (リス : le riz, と: et, おにぎり: onigiris, を: objet, たべました: j'ai mangé). Syntaxe : ``…~nom と nom~…''.

On l'utilise aussi pour dire que l'on fait quelque chose avec une autre personne : dans ce cas-là, on écrit ``…~une~personne \term{と}~…''.

\subsection{Forme longue intérogative}

Pour poser une question, on va former une phrase déclarative finissant par です ou par un verbe, et ajouter la terminaison か.

\paragraph{Compteurs}

Une phrase déclarative comporte très souvent l'information que l'on souhaite avoir (si on souhaite demander l'âge, la réponse est du style ``mon âge est 22 ans''), alors comment peut-on former une phrase déclarative pour poser cette question?

On va remplacer l'information que l'on souhaite connaître par un compteur, qui va permettre à l'interlocuteur de deviner ce qu'on l'on veut savoir. On utilise なに pour quelque chose qui n'est pas comptable, et なん pour quelque chose qui l'est.
Pour une liste des compteurs, voir \cref{sec:compteurs}.

Par exemple, puisque ``Je suis étudiant en Japonais'' se dit:
\begin{center}
 せんこうは\term{日本ご}です
\end{center}
on peut poser la question ``Quelle est ta filière ?'':
\begin{center}
 せんこうは\term{なん}です\term{か}
\end{center}

Souvent, on précise le type de réponse que l'on attend; par exemple si on attend une heure, on utilise なんじ, なんさい pour un âge, etc \cf{sec:compteurs}.

\subsection{Verbes}

Comme on l'a vu précédemment, une phrase doit se terminer par です ou par un verbe.
On retrouve deux groupes de verbes : les verbes en る, et les verbes en う.

Les verbes possèdent une `base', une `forme du dictionnaire', ainsi que plusieurs conjugaisons.

\subsubsection{Présent}

Les る-verbes portent ce nom parce que leur forme du dictionnaire se forme en ajoutant le suffixe る à leur base.
Par exemple, la base \ruby{食}{た}べ (manger) devient la forme du dictionnaire 食べる.
Pour former leur présent, on va ajouter à la base les suffixes ます pour l'affirmatif, et ません pour le négatif \cf{tab:ru-verbes-present}.

\begin{table}[h]
 \centering
 \caption{Exemples de る-verbes au présent}
 \label{tab:ru-verbes-present}
 \begin{tabular}{llrrr}
  Base & & Dictionnaire & Affirmatif & Négatif \\
  \hline
  \ruby 食{た}べ & manger & 食べ\term{る} & 食べ\term{ます} & 食べ\term{ません} \\
  \ruby 寝{ね} & dormir & 寝\term{る} & 寝\term{ます} & 寝\term{ません} \\
  \ruby 起{お}き & se lever & 起き\term{る} & 起き\term{ます} & 起き\term{ません} \\
  \ruby 見{み} & regarder & 見\term{る} & 見\term{ます} & 見\term{ません} \\
 \end{tabular}
\end{table}

Les う-verbes portent ce nom parce que leur forme du dictionnaire se termine par le son /u/.
La syllabe portant le son /u/ se déplace pour utiliser la voyelle /i/ (mais la même consonne), puis on ajoute les mêmes terminaisons que précédemment \cf{tab:u-verbes-present}.

\begin{table}[h]
 \centering
 \caption{Exemples de う-verbes au présent}
 \label{tab:u-verbes-present}
 \begin{tabular}{lrrr}
   & Dictionnaire & Affirmatif & Négatif \\
  \hline
  boire & \ruby 飲{の}\term{む} & 飲\term{みます} & 飲\term{みません} \\
  lire & \ruby 読{よ}\term{む} & 読\term{みます} & 読\term{みません} \\
  parler & \ruby 話{はな}\term{す} & 話\term{します} & 話\term{しません} \\
  écouter & \ruby 聞{き}\term{く} & 聞\term{きます} & 聞\term{きません} \\
  aller & \ruby 行{い}\term{く} & 行\term{きます} & 行\term{きません} \\
  rentrer à la maison & \ruby 帰{かえ}\term{る} & 帰\term{ります} & 帰\term{りません} \\
 \end{tabular}
\end{table}

On remarque que le verbe \ruby 帰{かえ}る est considéré comme un う-verbe, alors qu'il se termine par る.
Les verbes finissant par /u~ru/, /a~ru/ ou /o~ru/ sont toujours des う-verbes, les verbes finissant par /i~ru/ ou /e~ru/ sont souvent des る-verbes (par exemple \ruby 寝{ね}る), mais il existe aussi des exceptions (dont \ruby 帰{かえ}る).

On retrouve aussi deux verbes irréguliers, する et くる \cf{tab:irregulier-verbes-present}.

\begin{table}[h]
 \centering
 \caption{Verbes irréguliers au présent}
 \label{tab:irregulier-verbes-present}
 \begin{tabular}{lrrr}
   & Dictionnaire & Affirmatif & Négatif \\
  \hline
  aller & \term{する} & \term{します} & \term{しません} \\
  venir & \term{くる} & \term{きます} & \term{きません} \\
 \end{tabular}
\end{table}

Pour décrire un élément futur, on écrit une phrase au présent, que l'on situe dans le futur grâce à une particule (par exemple に).

Pour proposer à quelqu'un de faire quelque chose, on peut utiliser la forme négative suivie de か:
\begin{cquote}{ }
 \ruby 昼{ひる}ご\ruby 飯{はん}を\ruby 食{た}べ\term{ませんか}。
 Est-ce que tu serais intéressé par manger ce midi avec moi ?
\end{cquote}


\subsubsection{Passé}

%TODO

\subsubsection{Suffixes}

\subsection{Adjectifs}\label{sec:adjectifs}

\section{Forme en て}

\section{Forme courte}



%TODO Not rewritten yet

Voir \autoref{tab:greetings}.

\begin{table}[h]
 \centering
 \caption{Greetings}
 \label{tab:greetings}
 \begin{tabular}{ll}
  おはよう & Good morning \\
  こんいちは & Good afternoon \\
  さよおなら & Goodbye \\
  おやすみ & Good night \\
  ありがとう & Thank you \\
  すみません & Excuse me \\
  いいえ & No \\
  はい & Yes \\
  ええ & Yes (relaxé, conversations) \\
  いってきます & I'll go and come back \\
  いってらっしゃい & Please go and come back \\
  ただいま & I'm home \\
  おかえり & Welcome home \\
  いただきます & Thank you for the meal (before) \\
  ごちそおさま & Thank you for the meal (after) \\
  はじめまして & How are you? \\
  よろしくおねがいします & Nice to meet you \\
  あの & Interjection ``excusez-moi'' (équivalent d'un hm) \\
  そうですか & I see / Is that so? \\
 \end{tabular}
\end{table}

\section{Particules}

Voir \autoref{tab:particules-inside} et \autoref{tab:particules-outside}.

\begin{table}[h]
 \centering
 \caption{Particules sur un verbe}
 \label{tab:particules-outside}
 \begin{tabular}{lll}
 か & Question & verbe か \\
 ね & N'est-ce pas? & verbe ね \\
 よ & Je t'assure que & verbe よ \\
 ては & Il est interdit de & phrase\textsubscript{te} は いけません \\
 ても & Il est autorisé de & phrase\textsubscript{te} も いいです \\
 てもか & Est-ce qu'il est autorisé de & phrase\textsubscript{te} も いいですか \\
 \end{tabular}
\end{table}

\chapter{Expressions}

\section{Pointer, désigner}

Pour chaque mot pour pointer, le Japonais a trois versions:
\begin{itemize}
 \item une version pour désigner quelque chose proche de la personne qui parle,
 \item une version pour désigner quelque chose proche de la personne qui écoute,
 \item une version pour désigner ce qui est loin des deux,
 \item une version pour demander lequel est désigné.
\end{itemize}
La liste des mots est présentée dans le tableau \cref{tab:pointer}.

\begin{table}[htp]
 \centering
 \caption{Mots pour pointer}
 \label{tab:pointer}
 \begin{tabular}{lllll}
  & Proche de moi & Proche de toi & Loin de nous & Intérogatif \\
 \cline{2-5}
 celui-ci & これ & それ & あれ & どれ (lequel) \\
 ce\dots & この & その & あの & どの (quel\dots) \\
 ici, là & ここ & そこ & あそこ & どこ (où) \\
  & & & & だれ (qui) \\
 \end{tabular}
\end{table}

\section{Événements}

\subsection{Fréquence}

Pour déterminer la fréquence d'un événement, on va utiliser des adverbes de fréquence, par exemple:
\begin{cquote}{ }
 \ruby 私{わたし}は\term{よく}テレビを\ruby 見{み}ます。
 Je regarde souvent la télévision.
 
 メアリーさんは\term{ときどき}\ruby 朝{あさ}ご\ruby 飯{はん}を\ruby 食{た}べません。
 Certaines fois, Marie ne mange pas de petit-déjeûné.
\end{cquote}

Ils se placent devant un groupe d'action, sans particule.

Voir \cref{tab:freq}.
La phrase doit avoir la même valeur (positive, négative) que l'adverbe correspondant.

\begin{table}[h]
 \centering
 \caption{Vocabulaire de la fréquence}
 \label{tab:freq}
 \begin{tabular}{lllr}
  まいにち & 毎日 & Tous les jours & (positif) \\
  よく &  & Souvent & (positif) \\
  ときどき &  & Rarement & (positif) \\
  あまり &  & Presque jamais & (négatif) \\
  ぜんぜん &  & Jamais & (négatif) \\
 \end{tabular}
\end{table}

\chapter{Dictionnaire}

\section{Personnalité, occupation, nationalité}

\subsection{Pays, nationalité, langue}

De manière générale, la nationalité se forme avec le nom du pays, suivi du suffixe \ruby 人{じん}; de manière générale, le nom de la langue se forme avec le nom du pays suivi de \ruby 語{ご} \cf{tab:nationalite}.

\begin{table}[htp]
 \centering
 \caption{Pays, nationalité et langue}
 \label{tab:nationalite}
 \begin{tabular}{llrr}
 Pays & & Nationalité & Langue \\
 \hline
 \ruby{日本}{にほん} & Japon & 日本人 & 日本語 \\
 フランス & France & フランス人 & フランス語 \\
 イギリス & Angleterre & イギリス人 & \ruby{英}{えい}語 \\
 アメリカ & Amérique & アメリカ人 & アメリカ英語 \\
 かんこく & Corée \\
 ちゅうごく & Chine \\
 \end{tabular}
\end{table}

De la même manière, on utilisera なん人 pour demander la nationalité d'une personne.

\subsection{Métier}

\section{Objets}

\subsection{Vocabulaire}

Voir \cref{tab:objets:vie}, \cref{tab:objets:nourriture}.

\begin{table}[h]
 \centering
 \caption{Objets de la vie courante}
 \label{tab:objets:vie}
 \begin{tabular}{lll}
  えんぴつ &  & Crayon \\
  かさ &  & Parapluie \\
  かばん &  & Sac \\
  くつ &  & Chaussures \\
  さいふ &  & Porte-monaie \\
  ジーンズ &  & Jeans \\
  じしょ &  & Dictionnaire \\
  じてんしゃ &  & Vélo \\
  しんぶん &  & Journal \\
  Tシャツ &  & T-shirt \\
  とけい &  & Orloge, montre \\
  ノート &  & Carnet de notes \\
  ペン &  & Stylo \\
  ぼうし &  & Chapeau, casquette \\
  ほん & 本 & Livre \\
 \end{tabular}
\end{table}

\begin{table}[h]
 \centering
 \caption{Nourriture}
 \label{tab:objets:nourriture}
 \begin{tabular}{lll}
  さかな &  & Poisson \\
  やさい &  & Légumes \\
  とんかつ &  & Poitrine de porc \\
  にく &  & Viande \\
  メニュー &  & Menu \\
 \end{tabular}
\end{table}

\subsection{Lieux}

Voir \cref{tab:lieux}.

\begin{table}[h]
 \centering
 \caption{Lieux}
 \label{tab:lieux}
 \begin{tabular}{lll}
  きっさてん &  & Café \\
  ぎんこう &  & Banque \\
  トイレ &  & Toilettes \\
  としょかん &  & Bibliothèque \\
  ゆうびんきょく &  & Bureau de poste \\
 \end{tabular}
\end{table}

\subsection{Position relative (gauche, droite\dots)}

Voir \cref{tab:position-relative}.

\begin{table}[h]
 \centering
 \caption{Position relative de deux objets}
 \label{tab:position-relative}
 \begin{tabular}{ll}
  みぎ & À droite de \\
  ひだり & À gauche de \\
  まえ & Devant \\
  うしろ & Derrière \\
  なか & À l'intérieur de \\
  うえ & Dessus \\
  した & Dessous \\
  ちかく & Proche de \\
  となり & À côté de \\
 \end{tabular}
\end{table}

\section{Verbes}

Voir \autoref{tab:verbes-u}, \autoref{tab:verbes-ru} et \autoref{tab:verbes-irréguliers}.

Il faut faire attention aux verbes ある (il y a quelque chose) et いる (il y a quelqu'un), avec lesquels on utilise la particule に pour désigner un endroit (au lieu de で), et la particule が à la place de は pour le sujet.

\begin{table}[h]
 \centering
 \caption{Verbes en /u/}
 \label{tab:verbes-u}
 \begin{tabular}{lll}
  いく & 行く & Aller quelque part \\
  かえる & 帰る & Revenir de quelque part \\
  きく & 聞く & Écouter \\
  のむ & 飲む & Boire \\
  はなす & 話す & Parler \\
  よむ & 読む & Lire \\
  あう & 会う & Rencontrer quelqu'un \\
  ある & & Il y a (quelque chose) \\
  かう & 買う & Acheter \\
  かく & 書く & Écrire \\
  とる & 撮る & Prendre \\
  まつ & 待つ & Attendre \\
  わかる & & Comprendre \\
  およぐ & 泳ぐ & Nager \\
  きく & 聞く & Demander à quelqu'un \\
  のる & 乗る & Monter dans un véhicule \\
  やる & & Faire, réaliser \\
  あそぶ & 遊ぶ & Jouer, s'amuser \\
  いそぐ & 急ぐ & Se dépêcher \\
  おふろにはいる & お風呂に入る & Prendre un bain \\
  かえす & 返す & Rendre (objet) \\
  けす & 消す & Éteindre, effacer \\
  しぬ & 死ぬ & Mourir \\
  すわる & 座る & S'assoir \\
  たつ & 立つ & Se lever \\
  たばこをすう & たばこを吸う & Fumer \\
  つかう & 使う & Utiliser \\
  てつだう & 手伝う & Aider \\
  はいる & 入る & Entrer \\
  もつ & 持つ & Porter, tenir \\
  やすむ & 休む & Être absent, se reposer \\
 \end{tabular}
\end{table}

\begin{table}[h]
 \centering
 \caption{Verbes en る}
 \label{tab:verbes-ru}
 \begin{tabular}{lll}
  おきる & 起きる & Se lever \\
  たべる & 食べる & Manger \\
  ねる & 寝る & Dormir, aller dormir \\
  みる & 見る & Voir, regarder \\
  いる & & Il y a (quelqu'un) \\
  でかける & 出かける & Sortir \\
  あける & 開ける & Ouvrir \\
  おしえる & 教える & Apprendre à quelqu'un \\
  おりる & 降りる & To get off \\
  かりる & 借りる & Emprunter \\
  しめる & 閉める & Fermer \\
  つける & & Allumer \\
  でんわをかける & 電話をかける & Appeler quelqu'un \\
  わすれる & 忘れる & Oublier, abandonner un objet \\
 \end{tabular}
\end{table}

\begin{table}[h]
 \centering
 \caption{Verbes irréguliers}
 \label{tab:verbes-irréguliers}
 \begin{tabular}{lll}
  くる & 来る & Aller \\
  する & する & Faire \\
  べんきょうする & 勉強する & Étudier (faire des études) \\
 \end{tabular}
\end{table}

\subsection{Conjugaison}

\paragraph{Conjugaison de です}
Pour une affirmation en général, on utilise です (qui n'est pas un verbe).
\begin{itemize}
 \item Affirmatif Présent: です
 \item Négatif présent: じゃなかったです
 \item Affirmatif Passé: でした
 \item Négatif passé: じゃなかったでした
\end{itemize}

\paragraph{Forme normale}
En règle générale, les verbes au présent sont composés en ajoutant les suffixes suivants:
\begin{itemize}
 \item Affirmatif présent: ます
 \item Négatif présent: ません
 \item Affirmatif passé: ました
 \item Négatif passé: ませんでした
\end{itemize}

Pour les verbes en る, on peut donc simplement remplacer る par le suffixe (ねる $\mapsto$ ねます\dots).

Pour les verbes en /u/, on remplace aussi le son /u/ en /i/ (く $\mapsto$ きます, む $\mapsto$ みます\dots).

Le verbe irrégulier する devient し et se conjugue comme un verbe en る: します, しました\dots

Le verbe irrégulier くる devient き et se conjugue comme un verbe en る: きます, きました\dots

\paragraph{Forme en ましょう}
En conjugant en suivant les règles de la forme normale, mais avec ましょう à la place de ます, on peut demander une action:
\begin{itemize}
 \item Allons manger: \dots たべましょう
 \item Veux-tu manger?: \dots たべましょうか
\end{itemize}

\paragraph{Forme en て}
La forme de て est très importante, et elle est relativement compliquée.

Pour les verbes en る:
\begin{itemize}
 \item る $\mapsto$ て
\end{itemize}

Pour les verbes en /u/:
\begin{itemize}
 \item う/つ/る $\mapsto$ って
 \item む/ぶ/ぬ $\mapsto$ んで
 \item く/ぐ $\mapsto$ いて
 \item す $\mapsto$ して
\end{itemize}
Exception: 行く (いく) devient 行って.

Pour les verbes irréguliers:
\begin{itemize}
 \item する $\mapsto$ して
 \item くる $\mapsto$ きて
\end{itemize}

%TODO: te-imasu: une habitude

\section{Adjectifs}

Voir \autoref{tab:adjectifs-i} et \autoref{tab:adjectifs-na}.

\begin{table}[h]
 \centering
 \caption{Adjectifs en い}
 \label{tab:adjectifs-i}
 \begin{tabular}{lll}
  あたらしい & 新しい & nouveau \\
  あつい & 暑い & chaud (météo) \\
  あつい & 熱い & chaud (objet) \\
  いそがしい & 忙しい & occupé \\
  おおきい & 大きい & grand \\
  おもしろい & 面白い & intéressant, drôle \\
  かっこいい & & joli \\
  こわい & 怖い & effrayant \\
  さむい & 寒い & froid (météo) \\
  たのしい & 楽しい & drôle \\
  ちいさい & 小さい & petit \\
  つまらない & & ennuyant \\
  ふるい & 古い & vieux (objet) \\
  むずかしい & 難しい & difficile \\
  やさひい & & facile (objet), gentil (personne) \\
  やすい & 安い & pas cher (objet) \\
 \end{tabular}
\end{table}

\begin{table}[h]
 \centering
 \caption{Adjectifs en な}
 \label{tab:adjectifs-na}
 \begin{tabular}{lll}
  きれい(な) & & Beau, propre \\
  げんき(な) & 元気 & En bonne santé \\
  しずか(な) & 静か & Silencieux \\
  すき(な) & 好き & Aimer \\
  だいすき(な) & 大好き & Adorer \\
  きらい(な) & 嫌い & Ne pas aimer \\
  だいきらい(な) & 大嫌い & Haïr \\
  にぎやか(な) & & Vivant \\
  ひま(な) & 暇 & Avoir du temps libre \\
  たいへん(な) & 大変 & Compliqué (situation difficule) \\
 \end{tabular}
\end{table}

\paragraph{Conjugaison des adjectifs en い} Pour la conjugaison, on enlève le い final.
\begin{itemize}
 \item Affirmatif présent: いです
 \item Négatif présent: くないです
 \item Affirmatif passé: かったです
 \item Négatif passé: くなかったです
\end{itemize}

À noter que いい et les adjectifs formés à partir de lui (かっこいい\dots) se conjuguent comme s'ils étaient écrits よい en négatif, ou au passé (affirmatif et négatif).

\paragraph{Conjugaison des adjectifs en な} Pour la conjugaison, on enlève le な final.
\begin{itemize}
 \item Affirmatif présent: です
 \item Négatif présent: じゃないです
 \item Affirmatif passé: でした
 \item Négatif passé: じゃなかったです
\end{itemize}

\end{document}
