\documentclass[a4paper,10pt,french,openany]{memoir}
\usepackage[utf8]{inputenc}
\usepackage{babel}

\usepackage{clovisai}

%opening
\title{Droit du travail}
\author{Notes prises par Ivan Canet}

\begin{document}

\maketitle
\tableofcontents

\section{Introduction}

\paragraph{Hiérarchie des sources du droit}
Les différentes sources de droit, de la plus importante à la moins importante.
\begin{itemize}
 \item La constitution (de 1958)
 \item Les traités internationaux ou européens
 \item Les lois (gérées par le gouvernement)
 \item Les ordonnances
 \item Les décrets (décrets du conseil d'état et décrets ministériels)
 \item Les arrêtés
 \item Les conventions et accords collectifs
 \item Les coutumes et les usages
 \item La jurisprudence (décisions de justice passées sur lesquelles les tribunaux peuvent fonder leurs nouvelles décisions)
\end{itemize}

On retrouve ensuite le droit d'entreprise:
\begin{itemize}
 \item Les accords de branche
 \item Les accords de l'entreprise
 \item Les contrats
\end{itemize}

\paragraph{SMIC}
Le Salaire Minimum Interprofessionel de Croissance vaut 10.15€/heure brut, mais ne s'applique pas à tous les métiers:
\begin{itemize}
 \item contrat d'apprentissage pour les moins de 26 ans
 \item mineurs de moins de 18 ans (80\% du SMIC)
 \item contrat de professionalisation
\end{itemize}

\paragraph{Code du travail}
Dans le code du travail, il y a deux grandes parties qui se succèdent, la première est la partie législative (votée par le gouvernement) et la seconde est la partie règlementaire (décrets du conseil d'état ou du ministre du travail, ou règlements provenant de différents fonctionnaires).

La durée légale maximum du travail en France est de 35 heures par semaine (depuis 2000 pour les entreprises de plus de 20 salariés, et 2002 pour les autres).

\section{Contrat}



\section{Relations entre salariés et employeurs}
\section{Règlementation}
\section{Évolutions récentes}

\end{document}
